\begin{frame}[noframenumbering,plain]
    \setcounter{framenumber}{1}
    \maketitle
\note{Уважаемые члены совета позвольте представить Вам доклад по моей диссертационной работе. (0.5-0.5)}
\end{frame}


\begin{frame}
    \frametitle{Содержание}
    \tableofcontents
\end{frame}
\note{
    Работа состоит из пяти глав, назавания которых представлены на данном слайде.

    \medskip
    В первой главе приводится обзор литературы по различным методам моделирования, формулируются основы интегративного подхода к моделированию, приводится информация о ряде разработанных методов.
    Вторая, третья и четвертая главы посвящены моделированию структуры и динамики нуклеосом и их комплексов. Пятая глава - моделированию структуры амилоидоподобных фибрилл. Дальнейшее изложение следует плану, приведенному на  данном слайде. Динамическое оглавление будет отображаться вверху слайда. (1-1.5)
}




\section{Введение}
\subsection{Цель и задачи работы}
\begin{frame}
    \frametitle{Цель и задачи работы}
    \textbf{Целью} данной работы является разработка интегративных подходов к построению и методов анализа структурно-динамических моделей больших ДНК-белковых комплексов и амилоидоподобных фибрилл.\\
\vspace{0.5cm}

    В работе поставлены и решены следующие основные \textbf{задачи}:
\begin{enumerate}
\justifying
  \item Разработать подходы к интегративному моделированию ДНК-белковых комплексов методом молекулярной динамики.


  \item На основе различных экспериментальных данных разработать интегративные подходы к моделированию и верификации моделей комплексов ДНК и белков.
  
  \item Для интегративного моделирования биомакромолекулярных комплексов разработать методы анализа экспериментальных данных по расщеплению ДНК гидроксильными радикалами.

   
  \item Разработать подходы и методы построения моделей нуклеосом с белками хроматина.

   \item Разработать и применить интегративные подходы по построению атомистических моделей амилоидоподобных фибрилл.
\end{enumerate}

\end{frame}
\note{
Целью данной диссертационной работы являлась разработка интегративных подходов к построению и методов анализа структурно-динамических моделей больших ДНК-белковых комплексов и амилоидоподобных фибрилл. \\
Для достижения данной цели был поставлен ряд задач. Данные задачи включали разработку подходов к моделированию на основе методов молекулярной динамики, огрубленного моделирования, моделирования с использованием различных экспериментальных ограничений и данных. В некоторых случаях необходимо было разработать также методы численной обработки экспериментальных данных. Применить разработанные подходы предплагалось для трех классах биомакромолекулярных систем - нуклеосомах, комплексах нуклесом с белками хроматина и амилиодоподобных фибрилл.
Более детально задачи изложены на слайде и в автореферате. (1.5-3)
}


\normalsize

\subsection{Положения, выносимые на защиту}
\begin{frame}
    \frametitle{Положения, выносимые на защиту}
    \scriptsize
    \begin{enumerate}
        \justifying
\item   Для построения структурно-динамических моделей сложных биомакромолекулярных комплексов концептуально обосновано применение новых интегративных подходов на основе сочетания методов атомистического и огрубленного молекулярного моделирования, методов учета разнородных экспериментальных данных рентгеноструктурного анализа, атомно-силовой и электронной микроскопии, футпринтинга ДНК, ИК-, КД-спектроскопии, измерений расстояний между флуоресцентными метками на основе эффекта Ферстеровского резонансного переноса энергии.
\item   С использованием разработанного интегративного подхода возможно создание атомистических моделей нуклеосом и комплексов нуклеосом с белками хроматина, при этом учет симметрии белковых комплексов позволяет значительно повысить точность построения молекулярных моделей на основе данных футпринтинга ДНК.
\item   Интегративное моделирование позволяет воспроизвести на атомистическом уровне функциональную динамику нуклеосом, важную с точки зрения эпигенетической регуляции функционирования генома, включая крупномасштабные конформационные перестройки структуры ДНК-белковых комплексов (углы входа-выхода ДНК в нуклеосоме, диффузия гистоновых хвостов вдоль ДНК), а также позволяет обнаружить новые моды динамической подвижности, связанные с изменением конформации ДНК, перестройкой взаимодействий гистоновых хвостов, деформацией глобулярных доменов гистонов.
\item   Разработанный подход интегративного моделирования применим для построения молекулярных моделей амилоидоподобных фибрилл, реконструкции укладки пептидов в фибриллах и установления связи между морфологией фибриллы и межмолекулярной укладкой пептидов.
    \end{enumerate}
\end{frame}
\normalsize
\note{
    На слайде представлены положения выносимые на защиту, с ними подробнее можно ознакомиться в автореферате. (0.5-3.5)
}
