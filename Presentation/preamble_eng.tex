\begin{frame}[noframenumbering,plain]
    \setcounter{framenumber}{1}
    \maketitle
\note{Уважаемые члены совета позвольте представить Вам доклад по моей диссертационной работе.}
\end{frame}


\begin{frame}
    \frametitle{Contents}
    \tableofcontents
\end{frame}
\note{
    Работа состоит из пяти глав, назавания которых представлены на слайде.

    \medskip
    В первой главе приводится обзор литературы по различным методам моделирования биолоигческих молекул, формулируются основы интегративного подхода к моделирования, приводится информация о ряде разработанных методов.
    Вторая, третья и четвертая глава посвящены применению различных методов моделирования для установления структуры и динамики нуклеосом и их комплексов. В пятой главе разработаны и применеы методы интегративного моделирования для определения структуры амилоидоподобных фибрилл. Дальнейшее изложение следует плану, приведенному на слайде.
}




\section{Intoduction}
\subsection{Goals and tasks}
\begin{frame}
    \frametitle{Goals and tasks}
    The \textbf{goal} of this work is development of integrative approaches for building and analyzing structural and dynamical models of large DNA-protein complexes and amyloid-like fibrils.\\
\vspace{0.5cm}

    Following \textbf{tasks} were addressed:
\begin{enumerate}
\justifying
  \item Development of integrative approaches for modeling DNA-protein complexes using molecular dynamics simulations.

  \item Development of approaches for modeling and validation of DNA and protein complexes based on various experimental data.
  
  \item Development of integrative modeling methods based on hydroxyl-radical footprinting data

  \item Development of approaches for modeling nucleosome complexes with chromatin proteins.

  \item Development and application of integrative approaches for modeling of amyloid-like fibrils.
  
\end{enumerate}

\end{frame}
\note{
Целью данной диссертационной работы являлась разработка интегративных подходов к построению и методов анализа структурно-динамических моделей больших ДНК-белковых комплексов и амилоидоподобных фибрилл. Задачи работы состояли в \\
разработке  подходов к интегративному моделированию ДНК-белковых комплексов методом молекулярной динамики,\\
разработке интегративных подходов к моделированию и верификации моделей комплексов ДНК и белков на оснвое различных экспериментальных данных,\\
разработке метоов анализа экспериментальных данных по расщеплению ДНК гидроксильными радикалами для интегративного моделирования биомакромолекулярных комплексов,\\
разработке подоходов и методов  построения моделей нуклеосом с белками хроматина,\\
разработке и применению интегративных подходов по построению атомистических моделей амилоидоподобных фибрилл.\\
Более подробно задачи изложены в автореферате.
}


\normalsize

\subsection{Defendable propositions}
\begin{frame}
    \frametitle{Defendable propositions (in Russian)}
    \scriptsize
    \begin{enumerate}
        \justifying
\item   Для построения структурно-динамических моделей сложных биомакромолекулярных комплексов концептуально обосновано применение новых интегративных подходов на основе сочетания методов атомистического и огрубленного молекулярного моделирования, методов учета разнородных экспериментальных данных рентгеноструктурного анализа, атомно-силовой и электронной микроскопии, футпринтинга ДНК, ИК-, КД-спектроскопии, измерений расстояний между флуоресцентными метками на основе эффекта Ферстеровского резонансного переноса энергии.
\item   С использованием разработанного интегративного подхода возможно создание атомистических моделей нуклеосом и комплексов нуклеосом с белками хроматина, при этом учет симметрии белковых комплексов позволяет значительно повысить точность построения молекулярных моделей на основе данных футпринтинга ДНК.
\item   Интегративное моделирование позволяет воспроизвести на атомистическом уровне функциональную динамику нуклеосом, важную с точки зрения эпигенетической регуляции функционирования генома, включая крупномасштабные конформационные перестройки структуры ДНК-белковых комплексов (углы входа-выхода ДНК в нуклеосоме, диффузия гистоновых хвостов вдоль ДНК), а также позволяет обнаружить новые моды динамической подвижности, связанные с изменением конформации ДНК, перестройкой взаимодействий гистоновых хвостов, деформацией глобулярных доменов гистонов.
\item   Разработанный подход интегративного моделирования применим для построения молекулярных моделей амилоидоподобных фибрилл, реконструкции укладки пептидов в фибриллах и установления связи между морфологией фибриллы и межмолекулярной укладкой пептидов.
    \end{enumerate}
\end{frame}
\normalsize
\note{
    На слайде представлены положения выносимые на защиту, с ними подробнее можно ознакомиться в автореферате.
}
