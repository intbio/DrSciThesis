\begin{frame}
    \frametitle{Ответы на замечания ведущей организации НИИ~<<ЧаВо>>}
    \begin{itemize}
        \item Замечание -- ответ
        \item Замечание -- ответ
        \item Замечание -- ответ
        \item Замечание -- ответ
        \item Замечание -- ответ
    \end{itemize}
\end{frame}

\begin{frame}
    \frametitle{Ответы на замечания оф. оппонента Иванова\,И.\,И}
    \begin{itemize}
        \item Замечание -- ответ
        \item Замечание -- ответ
        \item Замечание -- ответ
        \item Замечание -- ответ
        \item Замечание -- ответ
    \end{itemize}
\end{frame}

\begin{frame}
    \frametitle{Ответы на замечания Петрова\,П.\,П}
    \begin{itemize}
        \item Замечание -- ответ
        \item Замечание -- ответ
        \item Замечание -- ответ
        \item Замечание -- ответ
        \item Замечание -- ответ
    \end{itemize}
\end{frame}

\begin{frame}
    \frametitle{Благодарности}
    \justifying
Автор выражает благодарность своим научным руководителям и консультантам, под руководством которых автору посчастливилось работать, М.П.Кирпи- чникову, А.Р.Хохлову, А.Р. Панченко, Д. Ландсману, П.Г. Халатуру, В.А. Иванову, всем соавторам своих научных работ и коллегам за плодотворное сотрудничество, в особенности, Г.А. Армееву, В.Б. Журкину, В.М. Студитскому, Х.-В. Чанг, Д.А. Гайкаловой, К.Ву, Х. Жао, А. Гончаренко, И. Драйзену, Е.-К. Шиллингер, О.С. Соколовой, А.В. Феофанову, Н.В. Малюченко, Е. Бондаренко, М. Валиевой, А. Любителеву и многим другим, коллективам кафедры физики полимеров и кристаллов физического факультета МГУ, кафедры биоинженерии биологического факультета МГУ, Национального Центра Биотехнологической Информации Национальных Институтов Здоровья за продуктивную рабочую атмосферу и обсуждение работы.

Автор выражает благодарность своей семье за поддержку, без которой написание этой работы не было бы возможным, и А.Д. Шайтану за помощь с версткой текста.
\note{Работы, описанные в диссертации были поддержаны рядом российских и международных грантов и стипендий, в том числе, грантами РНФ, РФФИ, стипендией сотрудничества России-США в области биомедицинских наук, стипендией Национальной медицинской библиотеки США, Немецким научно-исследовательским обществом. В работе активно использовалось  оборудование Центра коллективного пользования сверхвысокопроизводительными вычислительными ресурсами МГУ имени М.В. Ломоносова.}
\end{frame}