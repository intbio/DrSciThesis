\begin{frame}[allowframebreaks]
    \frametitle{Ответы на замечания офиц. оппонента С.В. Разина}
    \begin{enumerate}
        \item автор позиционирует свою работу как прежде всего методическую (о чем свидетельствует, в частности, формулировка целей работы). Мне кажется, что фундаментальное значение сделанных наблюдений является не менее важным, чем методические разработки. 
        \item Выводы работы могли бы быть сформулированы более коротко 
        \item В диссертации встречаются некоторые неудачные фразы, ошибочные утверждения и невычитанные опечатки 
        \item на странице 14 автор пишет: “Достоверность полученных результатов обеспечивается их публикацией в рецензируемых журналах международного уровня с высокими импакт-факторами”. С точки зрения рецензента, достоверность результатов обеспечивается правильной стратегией постановки экспериментов и наличием необходимых контролей. Публикация же результатов в престижных журналах может только подтвердить, но никак не обеспечить их достоверность.
        \item На стр. 82 автор пишет о методе Hi-C, ссылаясь на публикацию Либермана-Аидена 2009 г: «Оригинальный протокол метода, описанный в статье 2009 года [148], позволял достичь разрешения 4kb и выявлял, соответственно, в трехмерной структуре генома компартменты A и B….”. В действительности, в цитированной работе разрешение составляло 100 kb. 
        \item На странице 237 автор пишет: “Эпигенетическими метками активных энхансеров являются, в частности, H3K27ac, H3K4me1, H3K27me3, и гистоновые варианты H3.3 и H2A.Z”. Рецензенту трудно согласиться с тем, что модификация H3K27me3 является маркером активных энхансеров. В стволовых клетках присутствуют бивалентные домены, где на энхансерах есть как H3K27ac, так и H3K27me3. Однако активация энхансера сопряжена с удалением H3K27me3, которая привлекает репрессорный комплекс Polycomb 1.
    \end{enumerate}
\end{frame}

\begin{frame}[allowframebreaks]
    \frametitle{Ответы на замечания офиц. оппонента Р.Г. Ефремова}
    \begin{enumerate}
        \item     Материал, изложенный в Главе 2, представляет из себя набор модулей, каждый из которых соответствует опубликованным автором одной или нескольким работам по конкретной тематике. Считаю такой формат не слишком удачным с точки зрения целостного восприятия материала, поскольку неизбежно возникают повторы, например, при описании природы и устройства нуклеосом, их роли и пр., изложении методических подходов и пр. 
        \item    Недостатком работы, посвященной моделированию нуклеосом (Глава 2), является излишне краткое, подчас неинформативное описание вычислительных процедур и ряда полученных результатов. Кроме того, мало внимания уделяется обсуждению погрешностей моделирования и возможной чувствительности результатов молекулярной динамики (МД) к выбору исходных конфигураций рассматриваемых сложнейших объектов. В частности, расчеты МД проведены для ряда систем, содержащих изменения в исходной кристаллографической модели нуклеосомы, но при этом автор не поясняет, как подобные модификации структуры влияют на результаты МД. Тем более, что представленные здесь же данные показывают, что итоговые выводы могут измениться при выборе другого варианта «вмешательства» в модель нуклеосомы. Наконец, при обсуждении некоторых результатов вместо их исчерпывающего описания даны ссылки на опубликованные статьи автора и на интернет-ресурсы, что затрудняет анализ диссертационной работы 
        \item    Как справедливо отмечает автор (и в этом – суть ИМ-подхода!), без привлечения в качестве ограничений экспериментальных данных компьютерное моделирование таких сложных систем как нуклеосомы не позволяет пока достичь надежных результатов. В связи с этим особое внимание следовало бы уделить именно вопросу уточнения конкретных результатов расчетов в зависимости от данных экспериментов. На мой взгляд, было бы полезно показать в сравнении, что дает стандартный расчет, а что – ИМ-подход 
        \item    Как и в Главе 2, материал Главы 3 изложен в большой степени независимыми блоками, соответствующими публикациям по конкретной тематике. В результате имеются повторы в тексте, что затрудняет чтение 
        \item     При описании работы оператора с программой HYDROID часто упоминаются моменты, критическим (!) образом способные повлиять на результат. Кроме того, встречаются операции, которые необходимо проводить в «ручном» режиме. В какой степени это влияет на результаты профилирования и насколько программное обеспечение может быть использовано сторонними специалистами? Это важно с точки зрения как точности расчета, так и масштабирования программы и ее внедрения в практику исследований коллег, работающих в предметной области. 
        \item    При построении стартовых моделей амилоидоподобных фибрилл автор использует большое число произвольно (практически «на глаз») выбранных преобразований бета-структурных тяжей. Критериями в данном случае служит число водородных связей, отсутствие стерических наталкиваний в системе, «плотность упаковки» тиофеновых фрагментов и т.д. Насколько адекватны подобные конфигурации? Времена полноатомного моделирования МД составляют всего 10 нс, что, конечно, недостаточно для уравновешивания системы. Были ли предприняты попытки моделирования МД с независимых стартов? В какой степени результаты чувствительны к выбору исходной конфигурации? Пробовали ли делать выбор оптимальных состояний, основываясь на оценках свободной энергии системы? 
        \item     При анализе паттернов водородных связей в фибриллах (Глава 5) речь идет лишь о взаимодействиях белок-белок, а водородные связи с растворителем и в целом параметры гидратации фибрилл не обсуждаются. Вместе с тем, эти эффекты могут иметь важное значение для сборки и стабильности пептидных агрегатов 
        \item   При обсуждении результатов моделирования (в частности, из пептидов EF-C) в качестве экспериментальных данных сравнения приводится лишь фотография профиля фибриллы, полученная методом атомно-силовой микроскопии (Рис. 5.37в). При этом автор утверждает, что «рассчитанная 2D картина дифракции фибрилл» согласуется с экспериментальными данными. Однако деталей такого согласия не приводится, при этом паттерны на Рис. 5.37в и 5.38г довольно сильно отличаются друг от друга.  
        \item    В Главе 3 дано, на мой взгляд, излишне подробное описание программы HYDROID – местами оно, по сути, представляет собой руководство пользователя. Целесообразно было бы эти сведения дать либо в виде приложения, либо просто сослаться на соответствующий интернет-ресурс (ссылки на него и так есть) 
        \item   Значения ряда физических величин приведены в разных единицах, например, нм и Å.
        \item  В Главе 5 встречаются фрагменты текста, по-видимому, переведенные с английского языка с  помощью машинных средств. В частности, об этом свидетельствуют выражения: «производственный МД прогон» (стр. 343, возможно, «калька» с productive MD run?), «тетрамер был помещен в коробку для моделирования» (стр. 382) и др. 
        \item  Подписи к некоторым рисункам (например, 2.13, 2.17, 2.21) недостаточно информативны, в них не расшифровываются все детали, необходимые для понимания изображенного.

        \item  К недостаткам работы относятся и некоторые погрешности оформления. Так, автор использует ряд неудачных, жаргонных и некорректных выражений, например: «межмолекулярная укладка пептидов» (стр. 14), «рентгеновская структура», «способность нуклеосом претерпевать определенные типы конформационных переходов разумно используется … белками» (стр. 90), «полностью завернутое состояние» (стр. 99), «сложное динамическое взаимодействие между посттрансляционными модификациями гистонов» (стр. 128), «кристаллографические ионы» (стр. 129), «продвинутые навыки в языке Python» (стр. 190), «взаимодействие между внутренней геометрией… и геометрией, наложенной на ДНК…» (стр. 215) и пр. На стр. 341 вместо термина «постоянная длина» автор, по-видимому, имеет в виду персистентную длину. 

    \end{enumerate}
\end{frame}

\begin{frame}[allowframebreaks]
    \frametitle{Ответы на замечания оф. оппонента Н.В. Бриллиантова}
    \begin{enumerate}
        \item На стр. 114 автор не приводит оценки радиуса гирации. Так как эти оценки предствляются нетривиальными в рассматриваемом случае, было бы крайне интересно увидеть эти оценки.
        \item На стр. 115 приводися утвеждение "длинные олигокатионы имеют тенденцию почти полностью ассоциироваться с высокозаряженной ДНК из-за увеличения свободной энергии при освобождении небольших одновалентных ионов". Указанное утверждение весьма не полно. По-видимому, автор неявно предполагает, что система имеет достаточно большой объем, чтобы энтропийный вклад осовободившихся ионов домининровал.
        \item Автор изучил поведение системы при (чрезвычайно) высоком содержании соли, равном 1М и не обнаружил разборки нуклеосом (стр. 114) в работе это объясняется недостаточным временем моделирования. Не може ли это быть следствием того, что эффективная экранировка электростатических взаимодействий подавлялась за счет образования ионных пар? Возможно, следовало бы проверить, наличие и концентрацию ионных пар и ее соответствие экспериментальным заначениям при такой концентрации соли. -- Ответ: использовались параметры статьи Y. Lou and B. Roux "Simulation of Osmotic Pressure in Concentrated Aqueous Salt Solutions" J. Phys. Chem. Lett. - параметризация воспроизводит осмотическое давление до концентрации 4-5 М
        \item Удивление вызывает то, что автор не цитирует ряд основополагающих работ по теории образования амилодных фибрилл, такие как работы Прузинера (Prusiner, S.B. 1991. Molecular biology of prion diseases. Science 252:1515, Prusiner, S.B. 1999 An introduction to prion biology and diseases. CSHL), Айгена (Eigen, M. 1996 Prionics or the kinetic bais of prion diseases. Biophys Chem 63:A1) или Джефри (Jeffrey, M., I.A. Goodbrand and C.B. Goodsir 1995 Pathology of the transmissible spongiform encephalopathies with special emphasis on ultrastructure. Micron 26:277). Полагаю, что среди 677 ссылок должно было найтись место и для этих работ.
        \item В общепринятой модели Мазеля-Новака (Masel, J., V. A.A. Jansen and M.A. Novak 1999 Quantifying the kinetic parameters of prion replication Biophys. Chem 77:139) утверждается, что фибриллы содержащие меньше, чем n мономеров (n около 6) - неустойчивы и сразу распадаются. Было бы интересно проверить это утверждение, используя методы диссертационной работы.
        \item    В главе 2 автор обсуждает влияние электростатического отталкивания на углы входа выхода ДНК из нуклеосомы, изученное им с помощью моделирования нулеосом в растворах с различной ионной силой. Данный эффект можно было бы охарактеризовать не только качественно, но и количественно, например, проанализировав зависимость средней конформации линкерной ДНК при различной ионной силе.
    \item  В главе 3 автором разработан оригинальный метод определения положения ДНК на нуклеосоме с точностью до одного нуклеотида путем анализа данных о расщеплении ДНК гидроксильными радикалами. Не совсем понятно, как этот метод будет работать, если в растворе будет присутствовать смесь нуклеосом, взаимодействующих с матрицей ДНК в нескольких различных положениях.
    \item   В главе 5 предложенный автором алгоритм реконструкции крупномасштабной морфологии амилоидоподобных фибрилл основан на ручном или полуавтоматическом конструировании первоначальных периодических укладок пептидов. На мой взгляд, не до конца исследованным остается вопрос о зависимости общей формы фибриллы от тонких деталей первоначальной периодической укладки пептидов, например, ориентаций боковых цепей аминокислот.
    \item   Работа не лишена ряда опечаток и неудачных словесных оборотов. 
    \item Автор безусловно злоупотребляет англицизмами.
    \item в секции 2.3.3 автор многократно использует термин "приворот". Согласно справке из словаря Даля, "приворот - способ привлечения внимания, любви колдовством ...". Из контекста понятно, что имеется в виду, и речь не идет о колдовстве, однако, автору следует быть внимательнее.





    \end{enumerate}
\end{frame}

\begin{frame}
    \frametitle{Благодарности}
    \justifying
Автор выражает благодарность оппонентам, своим научным руководителям и консультантам, под руководством которых автору посчастливилось работать, М.П.Кирпичникову, А.Р.Хохлову, А.Р. Панченко, Д. Ландсману, П.Г. Халатуру, В.А. Иванову, всем соавторам своих научных работ и коллегам за плодотворное сотрудничество, в особенности, Г.А. Армееву, В.Б. Журкину, В.М. Студитскому, Х.-В. Чанг, Д.А. Гайкаловой, К.Ву, Х. Жао, А. Гончаренко, И. Драйзену, Е.-К. Шиллингер, О.С. Соколовой, А.В. Феофанову, Н.В. Малюченко, Е. Бондаренко, М. Валиевой, А. Любителеву и многим другим, коллективам кафедры физики полимеров и кристаллов физического факультета МГУ, кафедры биоинженерии биологического факультета МГУ, Национального Центра Биотехнологической Информации Национальных Институтов Здоровья за продуктивную рабочую атмосферу и обсуждение работы.

Автор выражает благодарность своей семье за поддержку, без которой написание этой работы не было бы возможным, и А.Д. Шайтану за помощь с версткой текста.
\note{\footnotesize{Прежде всего хочу поблагодарить оппонентов за внимательных анализ работы и ценные замечания. Хочу выразить благодарность своему научному консультанту М.П. Кирпичиникову за ценные советы и создание благориятных условий для работы над диссертацией, своим научным руководителям и консультантам, под руководством которых мне посчастливилось работать в течение своей научной жизни, А.Р.Хохлову, А.Р. Панченко, Д. Ландсману, П.Г. Халатуру, В.А. Иванову, всем соавторам своих научных работ и коллегам за плодотворное сотрудничество, в особенности, Г.А. Армееву, В.Б. Журкину, В.М. Студитскому, Х.-В. Чанг, Д.А. Гайкаловой, К.Ву, Х. Жао, А. Гончаренко, И. Драйзену, Е.-К. Шиллингер, О.С. Соколовой, А.В. Феофанову, Н.В. Малюченко, Е. Бондаренко, М. Валиевой, А. Любителеву и многим другим, коллективам кафедры физики полимеров и кристаллов физического факультета МГУ, кафедры биоинженерии биологического факультета МГУ, Национального Центра Биотехнологической Информации Национальных Институтов Здоровья за продуктивную рабочую атмосферу и обсуждение работы. Работы, описанные в диссертации были поддержаны рядом российских и международных грантов и стипендий, в том числе, грантами РНФ, РФФИ, стипендией сотрудничества России-США в области биомедицинских наук, стипендией Национальной медицинской библиотеки США, Немецким научно-исследовательским обществом. В работе активно использовалось  оборудование Центра коллективного пользования сверхвысокопроизводительными вычислительными ресурсами МГУ имени М.В. Ломоносова.
Отдельное спасибо хочу выразить своей семье за поддержку, без которой написание этой работы не было бы возможным, и А.Д. Шайтану за помощь с версткой текста.}}

\end{frame}