\section{Заключение}
\subsection{Обсуждение результатов}
\begin{frame}
    \frametitle{Обсуждение результатов}
\begin{itemize}
 \item Метод МД является мощным инструментом, однако его применение ограничено вычислительными ресурсами и качеством параметризации моделей.
  Для преодоления ограничений можно использовать интегративное моделирование.

\item Логика построения интегративных подходов следующая. При моделировании структурных элементов и параметров системы, на структуру и значения которых может сильно повлиять неточность силовых полей, необходимо дополнительно привлекать экспериментальные данные. Вместе с тем, моделирование конформационно устойчивых структурных элементов можно производить на основании данных силовых полей. 

\item В интегративном моделировании весьма полезным походом является использования соображений структурной симметрии, мультимасштабный подоход с комбинацией атомистического и огрубленного представлений молекулярных систем.

\note{Перейдем к обсуждению результатов и заключению диссертационной работы. В резуьтате работы ... (1.5-36)}
\end{itemize}


\end{frame}

\subsection{Выводы работы}
\begin{frame}%[allowframebreaks]
    \frametitle{Выводы работы}

\begin{enumerate}
\justifying
\scriptsize
%\tiny
% \fontsize{5}{6}
\item Разработан интегративный подход и программные решения для мультимасштабного моделирования комплексов ДНК и белков, в котором используются различные данные экспериментов по ДНК футпринтингу, электронной микроскопии, FRET-микроскопии и комбинированное представление ДНК в атомистическом и в динуклеотидном приближении.



  \item В микросекундном временном диапазоне определены параметры крупномасштабных функциональных конформационных изменений структуры нуклеосом: вычислен ансамбль конформаций линкерной ДНК, установлено влияние электростатического отталкивания на геометрию сегментов линкерной ДНК, определено характерное время диссоциации концов нуклеосомальной ДНК от гистонового октамера (10 мкс), установлены флуктуационные структурные механизмы образования дефектов кручения ДНК, переключения конформаций гистоновых хвостов.



  \item Обработка экспериментальных данных по расщеплению ДНК гидроксильными радикалами (футпринтинга ДНК) позволила вычислить вероятность расщепления ДНК для каждого нуклеотида. Показано, что профили расщепления ДНК гидроксильными радикалами в нуклеосоме мало зависят от последовательности ДНК, и определяются в основном позиционированием ДНК на нуклеосоме. Предложен алгоритм точного определения положения ДНК в нуклеосоме по данным футпринтинга для двух цепей ДНК на основании положения оси псевдосимметрии.

\end{enumerate}
\end{frame}

\begin{frame}%[allowframebreaks]
    \frametitle{Выводы работы (продолжение)}

\begin{enumerate}
\justifying
\scriptsize
%\tiny
% \fontsize{5}{6}

  \item С помощью методов интегративного моделирования были установлены структуры и параметры взаимодействий комплексов нуклеосом с РНК полимеразами, белком CENP-C, гистоном H1, белками комплекса FACT. Установлено, что в положении +49 после входа в нуклеосому РНК полимераза II может формировать компактный комплекс с нуклеосомой, в котором контакты гистонов с ДНК сохраняются по обе стороны активного центра. В модели центромерной нуклеосомы дрожжей определено положение ДНК и установлено, что белок CENP-C взаимодействует с нуклеосомой в районе 20 нуклеотидов от центра симметрии нуклеосомы. Предложены модели конформации линкерных сегментов ДНК при связывании гистона H1.  Установлены амплитуды конформационной подвижности ДНК в нуклеосомах при связывании с комплексом FACT.
  

 \item Разработанные подходы построения моделей амилоидоподобных фибрилл позволили изучить связь взаимного расположения пептидов в фибриллярных структурах с крупномасштабной морфологией амилоидоподобных фибрилл и таким образом установить структурную организацию филаментов на основе диблок олигомеров кватертиофена и пептида $(Thr-Val)_3$, а также на основе фрагмента белка gp120. Установлено, что рассмотренные амилоидоподобные фибриллы могут образовываться на основе двух взаимодействующих бета-листов, которые формируют либо плоскую ленту, либо левозакрученную ленту с периодом 24-30 нм.

\end{enumerate}
\end{frame}

% \begin{frame}
%     \frametitle{Научная новизна}
%     \begin{itemize}
%         \item Впервые реализован \dots
%         \item Разработана программа \dots
%         \item Впервые проведён анализ \dots
%         \item Предложена схема \dots
%     \end{itemize}
% \end{frame}
% \note{
%     Проговаривается вслух научная новизна
% }

% \begin{frame}
%     \frametitle{Научная и практическая значимость}
%     \begin{itemize}
%         \item Получены выражения для \dots.
%         \item Определены условия \dots.
%         \item Разработаны устройства \dots.
%     \end{itemize}
% \end{frame}
% \note{
%     Проговариваются вслух научная и практическая значимость
% }

% \begin{frame}
%     \frametitle{Свидетельство о регистрации программы}
%     \begin{figure}[h]
%         \centering
%         \includegraphics[height=0.7\textheight]{registration}
%     \end{figure}
% \end{frame}
% \note{
%     Получено свидетельство о регистрации разработанной программы \textsc{Hello~world™}.
% }

% \begin{frame}
%     \frametitle{Акт о внедрении}
%     \begin{figure}[h]
%         \centering
%         \fbox{
%             \begin{minipage}[t]{0.4\linewidth}
%                 \includegraphics[width=\linewidth]{implementation}
%             \end{minipage}
%         }
%     \end{figure}
% \end{frame}
% \note{
%     Получен акт о внедрении.
% }
\subsection{Основные публикации}
\begin{frame}[allowframebreaks] % публикации на одной странице
% \begin{frame}[t,allowframebreaks] % публикации на нескольких страницах
    \frametitle{Основные публикации}
% \nocite{hada_histone_2019,bass_effect_2019,armeev_linking_2019,shaytan_structural_2018,gorkovets_joint_2018,xiao_molecular_2017,shaytan_hydroxyl-radical_2017,gribkova_investigation_2017,el_kennani_ms_histonedb_2017,chertkov_dual_2017,armeev_modeling_2016,armeev_nucleosome_2016,biswas_genomic_2016,draizen_histonedb_2016,lyubitelev_structure_2016,shaitan_dynamics_2016,shaytan_coupling_2016,shaytan_trajectories_2016,valieva_large-scale_2016,armeev_conformational_2015,armeev_molecular_2015,frank_direct_2015,gaykalova_structural_2015,goncearenco_structural_2015,shaytan_nucleosome_2015,bozdaganyan_comparative_2014,chang_analysis_2014,kasimova_voltage-gated_2014,nishi_physicochemical_2014,sokolova_genome_2014,yolamanova_peptide_2013,shaitan_influence_2013,orekhov_calculation_2012,shaytan_self-assembling_2011,shaytan_self-organizing_2011}
% \renewcommand*{\bibfont}{\normalfont\scriptsize}
% %\renewcommand\bibliographytypesize{\small}
%     \ifnumequal{\value{bibliosel}}{0}{
%         \insertbiblioauthor
%     }{
%         \printbibliography%
%     }

\setbeamertemplate{enumerate items}[default]
\setbeamercolor*{enumerate item}{fg=black}
\scriptsize
\begin{enumerate}
 \justifying

 \item Histone Octamer Structure Is Altered Early in ISW2 ATP-Dependent Nucleosome Remode-ling / A. Hada, S. K. Hota, J. Luo, Y.-c. Lin, S. Kale, A. K. Shaytan, S. K. Bhardwaj, [et al.] // Cell Reports. –– 2019. –– July 2. –– Vol. 28, no. 1. –– 282––294.e6. –– DOI: 10.1016/j.celrep.2019. 05.106. –– \textbf{IF WoS 7.7} - (2,3/0,3).
\item The Effect of Oncomutations and Posttranslational Modifications of Histone H1 on Chroma-tosome Structure and Stability / M. V. Bass, G. A. Armeev, K. V. Shaitan, A. K. Shaytan // Moscow University Biological Sciences Bulletin. –– 2019. –– July. –– Vol. 74, no. 3. –– P. 121––126. –– DOI: 10.3103/S0096392519030015. –– \textbf{IF RINC 0.76} - (0,7/0,3).
\item  Linking Chromatin Composition and Structural Dynamics at the Nucleosome Level / G. A. Armeev, A. K. Gribkova, I. Pospelova, G. A. Komarova, A. K. Shaytan // Current Opinion in Structural Biology. –– 2019. –– June. –– Vol. 56. –– P. 46––55. –– DOI: 10.1016/j.sbi. 2018.11.006. –– \textbf{IF WoS 6.908} - (1,2/0,6).
\item  Structural Interpretation of DNA–Protein Hydroxyl-Radical Footprinting Experiments with High Resolution Using HYDROID / A. K. Shaytan, H. Xiao, G. A. Armeev, D. A. Gaykalova, G. A. Komarova, C. Wu, V. M. Studitsky, D. Landsman, A. R. Panchenko // Nature Protocols. –– 2018. –– Nov. –– Vol. 13, no. 11. –– P. 2535––2556. –– DOI: 10.1038/s41596-018-0048-z. –– \textbf{IF WoS 11.334} - (2,5/2,2).
\item  Joint Effect of Histone H1 Amino Acid Sequence and DNA Nucleotide Sequence on the Structure of Chromatosomes: Analysis by Molecular Modeling Methods / T. K. Gorkovets, G. A. Armeev, K. V. Shaitan, A. K. Shaytan // Moscow University Biological Sciences Bulletin. –– 2018. –– Apr. –– Vol. 73, no. 2. –– P. 82––87. –– DOI: 10 . 3103 / S0096392518020025. –– \textbf{IF RINC 0.76} - (0,7/0,3).
\item  Molecular Basis of CENP-C Association with the CENP-A Nucleosome at Yeast Centromeres / H. Xiao, F. Wang, J. Wisniewski, A. K. Shaytan, R. Ghirlando, P. C. FitzGerald, Y. Huang, [et al.] // Genes \& Development. –– 2017. –– Oct. 1. –– Vol. 31, no. 19. –– P. 1958––1972. –– DOI: 10.1101/gad.304782.117. –– \textbf{IF WoS 9.527} - (1,7/0,3).
\item  Hydroxyl-Radical Footprinting Combined with Molecular Modeling Identifies Unique Features of DNA Conformation and Nucleosome Positioning / A. K. Shaytan, H. Xiao, G. A. Armeev, C. Wu, D. Landsman, A. R. Panchenko // Nucleic Acids Research. –– 2017. –– Sept. 19. –– Vol. 45, no. 16. –– P. 9229––9243. –– DOI: 10.1093/nar/gkx616. –– \textbf{IF WoS 11.501} - (1,7/1,5).
\item  Gribkova, A. K. Investigation of Histone-DNA Binding Energy as a Function of DNA Unwrapping from Nucleosome Using Molecular Modeling / A. K. Gribkova, G. A. Armeev, A. K. Shaytan // Moscow University Biological Sciences Bulletin. –– 2017. –– July. –– Vol. 72, no. 3. –– P. 142––145. –– DOI: 10.3103/S009639251703004X. –– \textbf{IF RINC 0.76} - (0,5/0,2).
\item  MS\_HistoneDB, a Manually Curated Resource for Proteomic Analysis of Human and Mouse Histones / S. El Kennani, A. Adrait, A. K. Shaytan, S. Khochbin, C. Bruley, A. R. Panchenko, D. Landsman, D. Pflieger, J. Govin // Epigenetics \& Chromatin. –– 2017. –– Dec. –– Vol. 10, no. 1. –– DOI: 10.1186/s13072-016-0109-x. –– \textbf{IF WoS 5.333} - (2,1/0,5).
\item  Dual Active Site in the Endolytic Transglycosylase Gp144 of Bacteriophage phiKZ / O. V. Chertkov, G. A. Armeev, I. V. Uporov, S. A. Legotsky, N. N. Sykilinda, A. K. Shaytan, N. L. Klyachko, K. A. Miroshnikov // Acta Naturae. –– 2017. –– Vol. 9, no. 1. –– P. 7. –– DOI: 10.32607/20758251-2017-9-1-81-87. –– \textbf{IF WoS 1.62} - (0,8/0,1).
\item  Modeling of the Structure of Protein–DNA Complexes Using the Data from FRET and Footprinting Experiments / G. A. Armeev, T. K. Gorkovets, D. A. Efimova, K. V. Shaitan, A. K. Shaytan // Moscow University Biological Sciences Bulletin. –– 2016. –– Jan. –– Vol. 71, no. 1. –– P. 29––33. –– DOI: 10.3103/S0096392516010016. –– \textbf{IF RINC 0.76} - (0,6/0,2).
\item  Armeev, G. A. Nucleosome Structure Relaxation during DNA Unwrapping: Molecular Dynamics Simulation Study / G. A. Armeev, K. V. Shaitan, A. K. Shaytan // Moscow University Biological Sciences Bulletin. –– 2016. –– July. –– Vol. 71, no. 3. –– P. 141––144. –– DOI: 10.3103/S0096392516030020. –– \textbf{IF RINC 0.76} - (0,5/0,2).
 \item  Genomic Profiling of Multiple Sequentially Acquired Tumor Metastatic Sites from an “Exceptional Responder” Lung Adenocarcinoma Patient Reveals Extensive Genomic Hetero-geneity and Novel Somatic Variants Driving Treatment Response / R. Biswas, S. Gao, C. M. Cultraro, T. K. Maity, A. Venugopalan, Z. Abdullaev, A. K. Shaytan, [et al.] // Molecular Case Studies. –– 2016. –– Nov. –– Vol. 2, no. 6. –– a001263. –– DOI: 10.1101/mcs.a001263. –– \textbf{IF WoS 1.750} - (3,1/0,5).
 \item  HistoneDB 2.0: A Histone Database with Variants—an Integrated Resource to Explore Histones and Their Variants / E. J. Draizen, A. K. Shaytan, L. Marino-Ramirez, P. B. Talbert, D. Landsman, A. R. Panchenko // Database. –– 2016. –– Vol. 2016. –– baw014. –– DOI: 10.1093/database/baw014. –– \textbf{IF WoS 2.593} - (1,2/0,6).
 \item  Structure and Functions of Linker Histones / A. V. Lyubitelev, D. V. Nikitin, A. K. Shaytan, V. M. Studitsky, M. P. Kirpichnikov // Biochemistry (Moscow). –– 2016. –– Mar. –– Vol. 81, no. 3. –– P. 213––223. –– DOI: 10.1134/S0006297916030032. –– \textbf{IF WoS 1.978} - (1,3/0,1).
 \item  Shaitan, K. V. The Dynamics of Irreversible Evaporation of a Water–Protein Droplet and the Problem of Structural and Dynamic Experiments with Single Molecules / K. V. Shaitan, G. A. Armeev, A. K. Shaytan // Biophysics. –– 2016. –– Mar. –– Vol. 61, no. 2. –– P. 177––184. –– DOI: 10.1134/S0006350916020172. –– \textbf{IF SJR 0.226} - (0,9/0,1).
 \item  Coupling between Histone Conformations and DNA Geometry in Nucleosomes on a Micro-second Timescale: Atomistic Insights into Nucleosome Functions / A. K. Shaytan, G. A. Armeev, A. Goncearenco, V. B. Zhurkin, D. Landsman, A. R. Panchenko // Journal of Molecular Biology. –– 2016. –– Jan. –– Vol. 428, no. 1. –– P. 221––237. –– DOI: 10.1016/j.jmb.2015.12.004. –– \textbf{IF WoS 5.04} - (2,0/1,8).
 \item  Trajectories of Microsecond Molecular Dynamics Simulations of Nucleosomes and Nucleo-some Core Particles / A. K. Shaytan, G. A. Armeev, A. Goncearenco, V. B. Zhurkin, D. Landsman, A. R. Panchenko // Data in Brief. –– 2016. –– June. –– Vol. 7. –– P. 1678––1681. –– DOI: 10.1016/j.dib.2016.04.073. –– \textbf{IF WoS 0.97} - (0,5/0,5).
 \item  Large-Scale ATP-Independent Nucleosome Unfolding by a Histone Chaperone / M. E. Valieva, G. A. Armeev, K. S. Kudryashova, N. S. Gerasimova, A. K. Shaytan, O. I. Kulaeva, L. L. McCullough, [et al.] // Nature Structural \& Molecular Biology. –– 2016. –– Dec. –– Vol. 23, no. 12. –– P. 1111––1116. –– DOI: 10.1038/nsmb.3321. –– \textbf{IF WoS 11.98} - (0,9/0,1).
 \item  Armeev, G. A. Conformational Flexibility of Nucleosomes: A Molecular Dynamics Study / G. A. Armeev, K. V. Shaitan, A. K. Shaytan // Moscow University Biological Sciences Bulletin. –– 2015. –– July. –– Vol. 70, no. 3. –– P. 147––151. –– DOI: 10.3103/S009639251- 5030025. –– \textbf{IF RINC 0.76} - (0,6/0,3).
 \item  Armeev, G. A. Molecular Dynamics Study of the Ionic Environment and Electrical Character-istics of Nucleosomes / G. A. Armeev, K. V. Shaitan, A. K. Shaitan // Moscow University Biological Sciences Bulletin. –– 2015. –– Oct. –– Vol. 70, no. 4. –– P. 173––176. –– DOI: 10.3103/ S0096392515040033. –– \textbf{IF RINC 0.76} - (0,5/0,2).
 \item  Direct Prediction of Residual Dipolar Couplings of Small Molecules in a Stretched Gel by Stochastic Molecular Dynamics Simulations: Direct Prediction of Residual Dipolar Couplings by Stochastic MD Simulations / A. O. Frank, J. C. Freudenberger, A. K. Shaytan, H. Kessler, B. Luy // Magnetic Resonance in Chemistry. –– 2015. –– Mar. –– Vol. 53, no. 3. –– P. 213––217. –– DOI: 10.1002/mrc.4181. –– \textbf{IF WoS 1.731} - (0,6/0,1).
 \item  Structural Analysis of Nucleosomal Barrier to Transcription / D. A. Gaykalova, O. I. Kulaeva, O. Volokh, A. K. Shaytan, F.-K. Hsieh, M. P. Kirpichnikov, O. S. Sokolova, V. M. Studitsky // Proceedings of the National Academy of Sciences. –– 2015. –– Oct. 27. –– Vol. 112,
no. 43. –– E5787––E5795. –– DOI: 10.1073/pnas.1508371112. –– \textbf{IF WoS 9.412} - (1,0/0,3).
 \item  Structural Perspectives on the Evolutionary Expansion of Unique Protein-Protein Binding Sites / A. Goncearenco, A. K. Shaytan, B. A. Shoemaker, A. R. Panchenko // Biophysical Journal. –– 2015. –– Sept. –– Vol. 109, no. 6. –– P. 1295––1306. –– DOI: 10.1016/j.bpj.2015. 06.056. –– \textbf{IF WoS 3.665} - (1,4/0,4).
 \item  Shaytan, A. K. Nucleosome Adaptability Conferred by Sequence and Structural Variations in Histone H2A-H2B Dimers / A. K. Shaytan, D. Landsman, A. R. Panchenko // Current Opinion in Structural Biology. –– 2015. –– June. –– Vol. 32. –– P. 48––57. –– DOI: 10.1016/j.sbi. 2015.02.004. –– \textbf{IF WoS 6.908} - (1,2/1,0).
 \item  Comparative Computational Study of Interaction of C60-Fullerene and Tris-Malonyl-C60- Fullerene Isomers with Lipid Bilayer: Relation to Their Antioxidant Effect / M. E. Bozdaga-nyan, P. S. Orekhov, A. K. Shaytan, K. V. Shaitan // PLoS ONE / ed. by C. M. Soares. –– 2014. –– July 14. –– Vol. 9, no. 7. –– e102487. –– DOI: 10.1371/journal. pone.0102487. –– \textbf{IF WoS 2.740} - (0,9/0,1).
 \item  Analysis of the Mechanism of Nucleosome Survival during Transcription / H.-W. Chang, O. I. Kulaeva, A. K. Shaytan, M. Kibanov, K. Kuznedelov, K. V. Severinov, M. P. Kirpichnikov, D. J. Clark, V. M. Studitsky // Nucleic Acids Research. –– 2014. –– Feb. –– Vol. 42,
no. 3. –– P. 1619––1627. –– DOI: 10 . 1093 / nar / gkt1120. –– \textbf{IF WoS 11.501} - (1,0/0,2).
 \item  Voltage-Gated Ion Channel Modulation by Lipids: Insights from Molecular Dynamics Simulations / M. A. Kasimova, M. Tarek, A. K. Shaytan, K. V. Shaitan, L. Delemotte // Biochimica et Biophysica Acta (BBA) - Biomembranes. –– 2014. –– May. –– Vol. 1838, no. 5. –– P. 1322––1331. –– DOI: 10.1016/j.bbamem.2014.01.024. –– \textbf{IF WoS 3.4} - (1,2/0,1).
 \item  Nishi, H. Physicochemical Mechanisms of Protein Regulation by Phosphorylation / H. Nishi, A. Shaytan, A. R. Panchenko // Frontiers in Genetics. –– 2014. –– Aug. 7. –– Vol. 5. –– DOI: 10.3389/fgene.2014. 00270. –– \textbf{IF WoS 3.789} - (1,2/0,4).
 \item  Genome Packaging in EL and Lin68, Two Giant phiKZ-like Bacteriophages of P. Aeruginosa / O. Sokolova, O. Shaburova, E. Pechnikova, A. Shaytan, S. Krylov, N. Kiselev, V. Krylov // Virology. –– 2014. –– Nov. –– Vol. 468––470. –– P. 472––478. –– DOI: 10.1016/j.virol.2014.09. 002. –– \textbf{IF WoS 2.464} - (0,8/0,1).
 \item  Peptide Nanofibrils Boost Retroviral Gene Transfer and Provide a Rapid Means for Concent-rating Viruses / M. Yolamanova, C. Meier, A. K. Shaytan, V. Vas, C. W. Bertoncini, F. Arnold, O. Zirafi, [et al.] // Nature Nanotechnology. –– 2013. –– Feb. –– Vol. 8, no. 2. –– P. 130––136. –– DOI: 10.1038/nnano.2012.248. –– \textbf{IF WoS 31.538} - (0,8/0,2).
 \item  Influence of Interionic Interactions on Functional State and Blocker Binding of Voltage-Gated Potassium Channels / K. V. Shaitan, O. S. Sokolova, A. K. Shaitan, M. A. Kasimova, V. N. Novoseletskii, M. P. Kirpichnikov // Moscow University Biological Sciences Bulletin. –– 2013. –– Mar. –– Vol. 68, no. 1. –– P. 8––14. –– DOI: 10.3103/S009639251301- 0057. –– \textbf{IF RINC 0.76} - (0,8/0,1).
 \item  Orekhov, P. S. Calculation of Spectral Shifts of the Mutants of Bacteriorhodopsin by QM/MM Methods / P. S. Orekhov, A. K. Shaytan, K. V. Shaitan // Biophysics. –– 2012. –– Mar. –– Vol. 57, no. 2. –– P. 144––152. –– DOI: 10.1134/S0006350912020170. –– \textbf{IF SJR 0.226} - (1,0/0,2).
 \item  Self-Assembling Nanofibers from Thiophene–Peptide Diblock Oligomers: A Combined Experimental and Computer Simulations Study / A. K. Shaytan, E.-K. Schillinger, P. G. Khalatur, E. Mena-Osteritz, J. Hentschel, H. G. Borner, P. Bauerle, A. R. Khokhlov // ACS Nano. –– 2011. –– Sept. 27. –– Vol. 5, no. 9. –– P. 6894––6909. –– DOI: 10.1021/ nn2011943. –– \textbf{IF WoS 13.7} - (1,8/1,0).
 \item  Self-Organizing Bioinspired Oligothiophene–Oligopeptide Hybrids / A. K. Shaytan, E.-K. Schillinger, E. Mena-Osteritz, S. Schmid, P. G. Khalatur, P. Bauerle, A. R. Khokhlov // Beilstein Journal of Nanotechnology. –– 2011. –– Sept. 5. –– Vol. 2. –– P. 525––544. –– DOI: 10.3762/bjnano.2.57. –– \textbf{IF WoS 2.44} - (2,3/2,0).


\end{enumerate}

\end{frame}
\note{
    Основные результаты по теме диссертации изложены в 35 статьях в рецензируемых научных изданиях, индексируемых в базах данных Web of Science, Scopus, RSCI. Зарегистрированы 1 патент и 1 про­ грамма для ЭВМ.
}
\normalsize
\begin{frame}
\frametitle{Зарегистрированные патенты и программы}
\setbeamertemplate{enumerate items}[default]
\setbeamercolor*{enumerate item}{fg=black}
\begin{enumerate}
    \justifying
  \item Заявка 2580006 Рос. федерация, МПК G 06 F 19/100. Способ скрининга потенциальных противоопухолевых препаратов ингибиторов FACT [Текст] / В. М. Студитский, О. И. Студитская, А. К. Шайтан (Российская Федерация). — No 2013132806/10 ; заявл. 16.07.2013 ; опубл. 27.01.2015, Бюл. No 3 ; приоритет 16.07.2013 (Рос. Федерация). — 10 с.
\item Свидетельство о гос. регистрации программы для ЭВМ. Программный комплекс реконструкции пространственной структуры белков и комплексов на основе карт электронной плотности низкого разрешения [Текст] / Д. Л. Шуров, А. К. Шайтан, Г. А. Армеев, Д. А. Турченков, В. Н. Блинов, М. П. Кирпичников, К. В. Шайтан. — No 2013614397 ; заявл. 13.05.2013 ; опубл. 17.07.2013, 2013614397 (Рос. Федерация).
\end{enumerate}
\end{frame}

% \begin{frame}
%     \frametitle{Участие в конференциях}
%     \begin{itemize}
%         \item Научная сессия МГУ, Москва 2013--2015;
%         \item \rom{24} Russian Conference (RuC 2014), Obninsk, Russia, 2014
%         \item \rom{7} International Conference (IAC 16), Busan, Korea,
%               2016;
%         \item \rom{28} Other Conference (AC 16), East Lansing, MI USA, 2016;
%         \item \dots
%     \end{itemize}
% \end{frame}
% \note{
%     Работа была представлена на ряде конференций.
% }

\begin{frame}[plain, noframenumbering] % последний слайд без оформления
    \begin{center}
        \Huge
        Спасибо за внимание!
    \end{center}
\end{frame}
