\chapter*{Введение}                         % Заголовок
\addcontentsline{toc}{chapter}{Введение}    % Добавляем его в оглавление

\newcommand{\actuality}{}
\newcommand{\progress}{}
\newcommand{\aim}{{\textbf\aimTXT}}
\newcommand{\object}{{\textbf\objectTXT}}
\newcommand{\tasks}{\textbf{\tasksTXT}}
\newcommand{\novelty}{\textbf{\noveltyTXT}}
\newcommand{\influence}{\textbf{\influenceTXT}}
\newcommand{\methods}{\textbf{\methodsTXT}}
\newcommand{\defpositions}{\textbf{\defpositionsTXT}}
\newcommand{\reliability}{\textbf{\reliabilityTXT}}
\newcommand{\probation}{\textbf{\probationTXT}}
\newcommand{\contribution}{\textbf{\contributionTXT}}
\newcommand{\publications}{\textbf{\publicationsTXT}}
\newcommand{\acknowledge}{\textbf{\acknowledgeTXT}}

{\actuality}
%\ifdefined\DISSER
%Понимание механизмов функционирования живых систем на молекулярном и надмолекулярном (супрамолекулярном) уровнях является одной из ключевых задач биологии XXI века. Такое понимание важно не только с фундаментальной точки зрения, но и имеет прямые практические приложения в области медицины, здравоохранения, биоинженерии и биотехнологии. Значение термина ``понимание'' в этой связи хорошо проиллюстрировать цитатой Р. Фейнмана ``То, чего я не могу сделать - я не понимаю''\footnote{Англ. What I cannot create - I do not understand.}. Ключевой областью науки, способствующей пониманию работы живых систем на молекулярном уровне, является направление \textit{структурной биологии}. Работы середины XX века Д. Уотсона, Ф. Крика, М. Уилкинса, Р. Франклин, М. Перутца, Д. Кендрю заложили основы нашего понимания устройства ДНК и белков на атомистическом уровне и дали толчок развитию структурной и молекулярной биологии, биохимии на долгие годы вперед. Методы рентгеноструктурного анализа, биомолекулярной ЯМР-спектроскопии, электронной дифракции и микроскопии позволили определить по состоянию на 2020 год более 150 тысяч структур различных макромолекул и их комплексов\footnote{\url{https://www.rcsb.org/stats}}. На основе данной структурной информации были расшифрованы многие ключевые механизмы работы биологических систем.

%Со временем, однако, стали понятны и ограничения классических методов структурной биологии в изучении структуры и механизмов работы биомакромолекул и их комплексов. Можно выделить три типа ограничений. Во-первых, это ограничения, связанные с динамикой исследуемых биомолекул. Наличие различных конформаций биологического объекта приводит к тому, что определить его структуру, а также структуру различных конформаций становится значительно сложнее. Динамика объекта с одной стороны мешает пробоподготовке (например, кристаллизации в методах РСА), с другой ухудшает разрешение получаемой структурной модели, которая во многих случаях будет уже описывать некоторое усредненное конформационное состояние. Во-вторых, методы структурной биологии с трудом применимы к большим биомакромолекулярным комплексам -- по состоянию на 2020 год 92\% структур депонированные в базе PDB\footnote{\url{https://www.rcsb.org/stats/distribution-molecular-weight-entity}} имеют молекулярный вес менее 75 кДа, тогда как небольшой биомакромолекулярный комплекс, нуклеосома, имеет молекулярный вес около 200 кДа. Данный факт связан опять же с трудностью пробоподготовки больших структур и наличием крупномасштабных динамических мод. В-третьих, по мере увеличения размеров биомакромолекулярных комплексов механизмы их работы становятся все более сложными, связанными с переходами между многими конформационными состояниями, на вероятность которых влияют различные факторы физической и химической природы. Таким образом, становится все сложнее делать выводы о механизмах функционирования биомакромолекулярных комплексов, даже если удается разрешить их структуру. 

%Весьма характерным примером в этом плане является история изучения хроматина в эукариотических клетках. С момента открытия нуклеосом (определения их состава и наблюдения методами электронной микроскопии) в 1974 году \cite{kornberg_chromatin_1974-1,olins_spheroid_1974} потребовалось более 20 лет, чтобы определить детальную структурную организацию нуклеосом методами РСА \cite{luger_crystal_1997}. Попытки установить структуры комплексов нуклеосом с белками хроматина долгое время не увенчивались успехом. Так, чтобы получить структуру нуклеосомы с  небольшим белком, гистоном H1, ушло еще около 20 лет \cite{zhou_structural_2015}. В последнее время, благодаря успехам криоэлектронной микроскопии, удается разрешать структуры нуклеосом с более крупными комплексами белков хроматина, например, комплексами ремоделирвоания нуклеосом \cite{willhoft_structure_2018}. Однако из-за их динамичности, в том числе полиморфизма структры и меняющегося состава субъединиц, механизмы их функционирования и регуляции остаются плохо понятными. На более крупных масштабах организации хроматина динамичность взаимодействий проявляется еще сильнее. В настоящее время популярной является гипотеза, согласно которой в структурной организации хроматина важную роль играют физические и статистические эффекты, свойственные мягкой материи (soft matter). Например, эффекты фазового разделения между жидкими каплями различного состава и свойства (liquid-liquid phase separation) \cite{zhang_liquidliquid_2019}. Традиционно используемые в структурной биологии подходы представления структурной информации в принципе плохо применимы к описанию такого рода эффектов структурирования биологической материи.

%В задачах изучения механизмов биологических процессов важным дополнением к экспериментальным методам структурной биологии служат методы компьютерного молекулярного моделирования, которые позволяют рассчитывать свойства биомакромолекул на основе моделей физических взаимодействий между отдельными атомами или группами атомов. Для моделирования биологических молекул данные методы активно развиваются с начала 1980-ых годов. В качестве примера успешного взаимодополнения методов структурной биологии и компьютерного моделирования можно привести первые работы по моделированию связывания кислорода с миоглобином \cite{case_dynamic_1986}. Взрывной рост возможностей компьютерных вычислений, в том числе суперкомпьютерных, в последние десятилетия привел к серьезному прогрессу в области компьютерного молекулярного моделирования. Стало возможным моделирование систем состоящих из сотен тысяч и миллионов атомов (частиц) на временах десятков микросекунд, а с использованием специализированных суперкомпьютеров и на миллисекундных временах. Стало возможным моделирование фолднига небольших белков, наблюдения переключения белков между различными конформационными состояниями в ответ на внешнее воздействие \textit{in silico} \cite{jensen_mechanism_2012}.  Однако, присутствуют и существенные ограничения в области применения методов компьютерного моделирования для изучения биомолекулярных систем. Во-первых, это ограничения связанные с ограниченной возможность по изучению временной эволюции моделируемых систем. Многие физиологически важные процессы на супрамолекулярном уровне (например, динамика и ремоделирование нуклеосом) зачастую находятся в субсекундном и секундном диапазоне времен, тогда как методы моделирования пока в лучшем случае позволяют просчитывать динамику в микросекундном диапазоне. Второе ограничение связано с качеством параметризации атом-атомных взаимодействий, так называемых силовых полей. Несмотря на значительные успехи в этой области, возможность моделировать процессы фолдинга белков, а в некоторых случаях, предсказывать белковые структуры \textit{ab initio}, качество силовых полей не позволяет точно решать многие задачи, связанные  с правильным учетом вероятностного/энергетического баланса между различными конформациями биомакромолекул. Известно, что современные силовые поля испытывают затруднения при описании неструктурированных белков (intrinsically disordered proteins), а для глобулярных белков модели нативного состояния оказываются чрезмерно стабильными, переоценивающими гидрофобные эффекты \cite{shaytan_free_2010,petrov_are_2014}.

%Таким образом, весьма актуальной остается проблема построения и изучения структурно-динамических моделей биомакромолекулярных комплексов, в особенности в интервале размеров и динамических времен, не поддающихся прямой характеризации методами классической структурной биологии. Для интервала времен и размеров молекулярных систем, поддающихся описанию методами атомистического молекулярного моделирования, актуальными остаются проблемы сопряжения результатов расчета с непосредственно измеримыми в экспериментах параметрами. Развитию подходов, нацеленных на решение вышеозначенных проблем, и их применению для ряда биомакромолекулярных систем посвящена настоящая диссертационная работа. При построении структурно-динамических моделей биомакромолекулярных комплексов актуальным также является использование не только экспериментальных данных высокой детальности и информационного содержания (получаемых методами РСА, ЯМР, криоЭМ), которые могут быть доступны в ограниченном объеме, но и различных экспериментальных данных низкого информационного содержания, получаемых в ходе биофизических, биохимических, функциональных, спектроскопических и других экспериментов. Для названия подходов, основанных на интеграции различных экспериментальных данных для построения структурно-динамических моделей, будем использовать термин \textit{интегративное моделирование}, следуя практике использования данного термина в англоязычной литературе \cite{braitbard_integrative_2019}. В данной диссертационной работе методы интегративного моделирования разрабатывались и применялись, в том числе для создания молекулярных моделей нуклеосом, комплексов нуклеосом с белками хроматина, амилоидоподобных фибрилл. Актуальность исследования нуклеосом и их комплексов напрямую связана с необходимостью понимания функционирования хроматина в клетках эукариотических организмов, что в свою очередь необходимо для понимания функционирования и развития живых организмов. Актуальность исследования амилоидоподобных фибрилл связана с их вовлеченностью в ряд биологических процессов, в том числе патологических, а также с возможностью использования эффектов самосборки пептидов в амилоидоподобные фибриллы для создания функциональных нанотехнологических и биотехнологических конструкций.
%\else
Понимание механизмов функционирования живых систем на молекулярном и супрамолекулярном уровнях является одной из ключевых задач биологии XXI века. Такое понимание важно не только с фундаментальной точки зрения, но и имеет прямые практические приложения в области медицины, здравоохранения, биоинженерии и биотехнологии. 
Особенностью биологических молекул является формирование ими сложных пространственных структур, которые могут перестраиваться в ходе различных функциональных процессов.
Исследование \textit{структурно-динамических} моделей биомакромолекул является одним из ключевых подходов к изучению механизмов функционирования живой материи.
Методы компьютерного молекулярного моделирования являются неотъемлемой частью этого подхода. Среди методов построения моделей важное место занимают методы, использующие экспериментальные данные рентгеноструктурного анализа (РСА), ядерно-магнитного резонанса (ЯМР), криоэлектронной микроскопии (криоЭМ). В настоящее время они сочетаются с методами молекулярной динамики (МД), Монте-Карло, позволяющих проводить \textit{компьютерные эксперименты}, имитационное моделирование молекулярных систем. В этой области существуют и ограничения, которые затрудняют исследования биологический систем значительных размеров таких, как большинство биомакромолекулярных комплексов. Сложность инетерпретации и самого получения данных РСА, ЯМР, криоЭМ возрастает по мере увеличения размера и конформационной подвижности исследуемой системы. По состоянию на 2020 год 92\% структур депонированные в базе PDB имеют молекулярный вес менее 75 кДа. Возможности исследования больших биомакромолекулярных комплексов (от 100 кДа и более) методами молекулярной динамики ограничиваются вычислительными возможностями и качеством параметризации атом-атомных взаимодействий (так называемых силовых полей). Несмотря на то, что благодаря прогрессу компьютерных технологий, методом МД стал возможен расчет систем, состоящих из сотен тысяч и миллионов атомов на временах десятков микросекунд (а с использованием специализированных суперкомпьютеров и на миллисекундных временах), временные масштабы многих физиологически важных процессов, включая важные моды динамики многих биомакромолекулярных комплексов, находятся пока за пределами возможностей моделирования в субсекундном и секундном диапазоне времен.  С ростом размера биомолекулярной системы неточности задания атом-атомных взаимодействий суммируются и сильнее отражаются на качестве воспроизведения ее динамики, в то время как количество динамических состояний увеличивается, а разница свободной энергии между ними уменьшается.
\textit{Преодоление вышеописанных ограничений с целью построения и анализа структурно-динамических моделей биомакромолекулярных комплексов является важной задачей.}
На современном этапе одним из основных путей решения данной задачи является разработка и совершенствование подходов, которые используют для построения моделей как расчет физических взаимодействий между атомами, так и различные экспериментальные данные. К таким экспериментальным данным могут относится как данные классических методов структурной биологии (данные РСА, ЯМР, криоЭМ), так и данные спектральных методов, различных видов микроскопии, биохимического анализа. 
%Методы работы с последним типом данных, например, расстояниями между молекулярными группами, вероятностью их взаимодействия с химическими агентами, формой молекулярного комплекса особенно актуальны ввиду относительной простоты их экспериментального получения.
%С развитием вычислительных технологий важное значение приобрели методы компьютерного молекулярного моделирования. Данные методы могут быть использованы, как для эффективной интерпретации экспериментальных данных, так и для расчетов структуры и динамики биомолекул на основе моделей физических взаимодействий между атомами. К последнему классу методов, относятся \textit{компьютерные эксперименты} на основе метода молекулярной динамики (МД).
%В настоящее время методом МД стал возможен расчет систем состоящих из сотен тысяч и миллионов атомов на временах десятков микросекунд, а с использованием специализированных суперкомпьютеров и на миллисекундных временах. Однако, в этой области также присутствуют существенные ограничения. Во-первых, временные масштабы многих физиологически важных процессов, включая важные моды динамики многих биомакромолекулярных комплексов, находятся за пределами возможностей моделирования в субсекундном и секундном диапазоне времен. Второе ограничение связано с качеством параметризации атом-атомных взаимодействий, так называемых, силовых полей. С ростом размера молекулярной системы неточность задания силовых полей обычно сильнее сказывается на качестве структурно-динамической модели.

%\fi

%степень разработанности темы исследования
{\progress} 
%\ifdefined\DISSER
%Обсуждение степени разработанности темы исследований структурировано ниже следующим образом - в начале обсуждается степень разработанности темы компьютерного молекулярного моделирования в целом, затем степень разработанности методов интегративного моделирования, далее обсуждаются степень разработанности конкретных тем, освещенных в диссертации по следующему плану: моделирование нуклеосом методом молекулярной динамики, моделирование на основе данных по расщеплению ДНК гидроксильными радикалами, интегративное моделирование комплексов нуклеосом и белков хроматина, моделирование амилоидоподобных фибрилл.

%Методы компьютерного молекулярного моделирования активно развиваются с середины 1940-годов и по настоящее время. Как отмечалось выше, достигнут существенный прогресс в возможностях моделирования биологических систем размерами в сотни тысяч и миллионы атомов на временах до десятков микросекунд и даже миллисекунд. Основными стратегическими направлениями дальнейших работ является увеличение доступных времен моделирования, благодаря совершенствованию программных и аппаратных технологий, а также совершенствование силовых полей для максимально реалистичной параметризации моделей биомакромолекул.

%Интегративное моделирование является достаточно широким классом методов и подходов. Задачи поиска структуры биомолекул, отвечающих заданным экспериментальным данным, решаются, в том числе и при построении моделей методами РСА, криоЭМ, ЯМР, но для определенного класса экспериментальных данных. Если в случае РСА и криоЭМ построение моделей может зачастую производится вручную путем вписывания координат атомов в электронную плотность, то интерпретация данных ЯМР уже требует компьютерных алгоритмов оптимизации структуры на основе более косвенных экспериментальных данных, например, расстояний между атомами. Поэтому программы используемые для реконструкции структур по данным ЯМР, вероятно, первыми начали включать в себя функциональность по использованию некоторых типов данных из других экспериментов, например, данных малоуглового рентгеновского и нейтронного рассеяния  \cite{schwieters_xplor-nih_2003}. Похожие методы оптимизации решаются при построении моделей белков по гомологии  \cite{sali_comparative_1993}. Существуют разработки по использованию карт электронной плотности низкого разрешения для направления структурных деформаций биомолекул при расчетах методами молекулярной динамики и оптимизационного моделирования \cite{mcgreevy_xmdff_2014,wriggers_conventions_2012}. Некоторыми группами разрабатываются программные платформы для более системного подхода к интегративному моделированию биомолекул с учетом данных различных биофизических и биохимических экспериментов \cite{russel_putting_2012}. Следует однако отметить, что в отличие от подходов атомистической молекулярной динамики, подходы интегративного моделирования являются достаточно разнообразными и зачастую заточены под моделирование определенного класса систем с использованием определенного класса экспериментальных данных. Их систематизацию можно проводить по крайней мере по трем характеристиками - (i) уровню представления модели (напр. атомистическое, огрубленное), (ii) способу оценки соответствия модели экспериментальным данным, включая возможное задание скоринговой функции, (iii) методам поиска структуры, соответствующей экспериментальным данным, включая алгоритмы оптимизации и задания варьируемых параметров. В этом плане имеющиеся программные продукты и платформы пока далеки от универсальности.

%Моделирование нуклеосом методами молекулярной динамики является достаточно активной темой исследований, количество публикаций по которой растет и приближается к сотне\footnote{Запрос к PubMed <<``molecular dynamics simulations'' nucleosome>> выдает 86 записей на момент написания диссертации.}. По мере роста производительности вычислительных ресурсов открываются новые возможности по исследованию динамики нуклеосом на все больших временах. В 2016 году нами были опубликованы результаты по микросекундному моделированию, в данной работе приведены уже результаты по моделированию на временах превышающих 10 микросекунд.

%%Оценка термодинамических параметров гидратации малых молекул с высокой точностью путем прямых расчетов методом молекулярной динамики в явном растворителе стала возможной в начале XXI века благодаря прогрессу в скорости вычислений и развитию методов так называемой вычислительной алхимии (расчетов систем с гамильтонианом взаимодействий между частями, зависящим от варьируемого параметра). С тех пор по этой теме было опубликовано достаточно много работ, однако, исследователями рассматривалась гидратация в объеме растворителя, а не вблизи поверхности воды.


%Методы футпринтинга ДНК гидроксильными радикалами развиваются с конца 1980-ых годов \cite{churchill_detection_1990}. Существенный вклад в понимание механизмов расщепления ДНК был внесен группой Т. Туллиуса \cite{balasubramanian_dna_1998}. Этой же группой были заложены основы цифровой обработки данных футпринтинга, реализованной в ряде программ \cite{shadle_quantitative_1997,das_safa_2005}. Однако по состоянию на середину второй декады XXI века программные разработанные ранее программные инструменты не поддерживались разработчиками, и их использование было серьезно затруднено, особенно с целью интеграции получаемых данных в комплексные пайплайны обработки, визуализации и моделирования.

%Интегративное моделирование структуры комплексов нуклеосом с белками хроматина долгое время было затруднительным из-за отсутствия структуры нуклеосомы высокого разрешения, которая была получена в 1997 году \cite{luger_crystal_1997}. Следующим важным шагом, упрощающим построение таких моделей, явились разработки методов описания конформации ДНК в пространстве координат взаимного расположения азотистых оснований с возможностью перехода из огрубленного приближения в атомистическое и оценкой энергии изгиба ДНК \cite{olson_dna_1998,lu_3dna_2003}. На данный момент возможности построения и анализа интегративных моделей нуклеосом с белками хроматина ограничены в основном доступностью качественных экспериментальных данных.

%Тема изучения амилоидоподобных фибрилл является весьма популярной тематикой исследований, в том числе методами молекулярного моделирования \cite{lu_understanding_2018}. Благодаря вовлеченности процессов амилоидизации в ряд болезней человека, существенная часть работ посвящена изучению определенных типов фибрилл, в частности фибрилл формируемых амилоидом $\beta$. Для фибрилл, представляющих чрезвычайный медицинский интерес, доступно большое количество экспериментальных данных, в том числе методами ЯМР, РСА \cite{sawaya_atomic_2007}.  Исследования других типов фибрилл являются не настолько активными, для них доступно меньше экспериментальных данных и актуальными являются подходы интегративного моделирования.


%\else
Методы молекулярного компьютерного моделирования зародились в 1940-ых годах с появлением первых вычислительных систем, однако активное их применение к биологическим молекулам стало возможным в 1970-ых -- 1980-ых годах. Активно стали развиваться как подходы к моделированию биомолекул методом МД, так и оптимизационные методы моделирования структуры на основе экспериментальных данных (главным образом РСА и ЯМР). Методы МД основаны на представлении молекулярной системы в виде набора атомов, взаимодействующих согласно законам классической механики Ньютона, и численном решении уравнений движения. На сегодняшний день разработана обширная методическая база для моделирования белков, нуклеиновых кислот, липидов методами молекулярной динамики, разработаны наборы функциональных форм и параметров, описывающих взаимодействия между атомами (например, семейства силовых полей CHARMM, AMBER, OPLS и др.), алгоритмы численного интегрирования уравнений движения, поддержания температуры, давления, использования параллельных суперкомпьютерных технологий, созданы программные комплексы для расчетов (например, AMBER, CHARMM, GROMACS, NAMD, PUMA и др.).
В то же время уровень развития методов МД, как с точки зрения доступных времен моделирования, так и с точки зрения качества силовых полей, не позволяет использовать их полностью независимым образом в отрыве от экспериментальных данных. Стандартной практикой для расчетов методом МД является использование в качестве стартовой конформации биомакромолекулярной системы структурной модели, полученной на основе данных РСА, ЯМР, криоЭМ. 
%В случае, если такая модель отсутвует, применение методов МД становится практически невозможным. Зачастую, такая ситуация возникает при изучении биомакромолекулярных комплексов. 

Важный вклад в построение структурных моделей биомакромолекулярных системы вносят комбинированные вычислительные подходы, которые, кроме данных РСА, ЯМР, криоЭМ, могут использовать различные экспериментальные данные, получаемые в ходе биофизических, биохимических, функциональных, спектроскопических и других экспериментов. Такого рода эксперименты могут предоставлять информацию о расстояниях между введенными метками в белке или ДНК (например, методы FRET, ЭПР), характере укладки белковой цепи (например, ИК-, КД-спектроскопия, дифракция рентгеновских лучей на фибриллах), реакционной доступности химических групп (методы футпринтинга, химического сшивания) и др. Такой подход, основанный на интеграции различных экспериментальных данных при построении структурно-динамических моделей, в литературе называется \textit{интегративным моделированием} \cite{braitbard_integrative_2019}\footnote{Ссылки на список литературы приведены цифрами, ссылки на список статей автора -- цифрами \ifdefined\DISSER с буквой A. \else со звездочкой.\fi}. Программы, используемые для реконструкции структур по данным экспериментов ЯМР, вероятно, первыми начали позволять использовать некоторые типы данных из других экспериментов, например, данные малоуглового рентгеновского и нейтронного рассеяния  \cite{schwieters_xplor-nih_2003}. Другим примером является использование в методах МД дополнительных членов потенциальной энергии, которые зависят от экспериментально полученных карт электронной плотности низкого разрешения методами электронной микроскопии или РСА. Такой подход позволят деформировать начальные структуры биомолекул для придания им экспериментально наблюдаемой формы \cite{mcgreevy_xmdff_2014}. Разрабатываются программные подходы, предоставляющие возможность одновременно использовать данные различных биофизических и биохимических экспериментов (например, карты электронной плотности и расстояния между группами атомов) \cite{russel_putting_2012}. В целом подходы интегративного моделирования могут различаться по (i) уровню представления модели (например, атомистическое или огрубленное представление модели), (ii) способу оценки соответствия модели экспериментальным данным (использование различных силовых полей или оценочных функций), (iii) методам поиска структуры, соответствующей экспериментальным данным, включая использование различных алгоритмов оптимизации, в том числе минимизации энергии, расчетов методом Монте-Карло, (iv) заданию типа варьируемых параметров (координаты атомов, групп атомов, обобщенные координаты реакции). В настоящее время универсальных подходов интегративного моделирования не существует, такие подходы требуют специальных методов и программных реализаций под конкретный тип моделируемых систем (например, комплексы белок-ДНК, амилоидоподобные фибриллы, мембранные системы и т.д.) и конкретный набор экспериментальных данных.  
%Особенно актуальным это становится при изучении  ``горячих'' областей и объектов современной биологии, связанных с молекулярными механизмами функционирования, например, генетического аппарата клетки.  
%Способы изучения струкутрно-динамических характеристик биомолекул и их комплексов можно разделить на две группы подходов. Первая группа подходов связана с получением большого количества однотипных экспериментальных данных и построения на их основе моделей структуры молекул путем решения оптимизационных задач. К таким подходам относятся классические методы структурной биологии: РСА, ЯМР, крио-ЭМ. Вторая группа подходов связана с молекулярным моделированием на основе физических моделей взаимодействий между атомами, компьютерными экспериментами. К данной группе относятся подходы МД, Монте-Карло и другие. Эти методы активно развиваются с середины 1940-годов и по настоящее время. Несмотря на существенный прогресс как экспериментальных, так и теоретических, расчетных подходов изучение многих крупных биомакромолекулярных систем находится за пределами возможностей данных методов. В то же время использование комбинированных вычислительно-экспериментальных подходов открывает определенные дополнительные возможности по изучению биомакромолекулярных комплексов. При построении структурно-динамических моделей биомакромолекулярных комплексов становится возможным дополнительно использовать различные экспериментальные данные низкого информационного содержания, получаемые в ходе биофизических, биохимических, функциональных, спектроскопических и других экспериментов. Такого рода эксперименты могут предоставлять информацию о расстоянии между введенными метками в белке или ДНК (например, методы FRET, ЭПР), характере укладки белковой цепи (например, ИК-, КД-спектроскопия, дифракция рентгеновских лучей на фибриллах), реакционной доступности химических групп (методы футпринтинга) и др. Для подходов, основанных на интеграции различных экспериментальных данных для построения структурно-динамических моделей, в литературе используется термин \textit{интегративное моделирование} \cite{braitbard_integrative_2019}. Программы используемые для реконструкции структур по данным ЯМР, вероятно, первыми начали включать в себя функциональность по использованию некоторых типов данных из других экспериментов, например, данных малоуглового рентгеновского и нейтронного рассеяния  \cite{schwieters_xplor-nih_2003}. Похожие методы оптимизационные задачи решаются при построении моделей белков по гомологии. Имеются также разработки по использованию карт электронной плотности низкого разрешения для направления структурных деформаций биомолекул при расчетах методами молекулярной динамики и оптимизационного моделирования \cite{mcgreevy_xmdff_2014}. Некоторыми группами ведутся работы по созданию программных платформ для интегративного моделирования, предоставляющих возможность одновременно использовать данные различных биофизических и биохимических экспериментов (например, карты электронной плотности и расстояния между группами атомов) \cite{russel_putting_2012}. Систематизацию таких подходов можно проводить по крайней мере по трем характеристиками - (i) уровню представления модели (например, атомистическое или огрубленное представление модели), (ii) способу оценки соответствия модели экспериментальным данным (использование различных силовых полей или оценочных функций), (iii) методам поиска структуры, соответствующей экспериментальным данным, включая использование различных алгоритмов оптимизации (минимизация, метод Монте-Карло и т.д.) и задания варьируемых параметров (координаты атомов, групп атомов, обобщенные координаты реакции). Следует однако отметить, что в отличие от подходов атомистической молекулярной динамики, универсальных подходов интегративного моделирования на данный момент не существует, такие подходы требуют специальных методов и программных реализаций под конкретный тип моделируемых систем (например, комплексы белок-ДНК, амилоидоподобные фибриллы, мембранные системы и т.д.) и конкретный набор экспериментальных данных.  
%Особенно актуальным это становится при изучении  ``горячих'' областей и объектов современной биологии, связанных с молекулярными механизмами функционирования, например, генетического аппарата клетки.  

%Моделирование нуклеосом методами молекулярной динамики является достаточно активной темой исследований, количество публикаций по которой растет и приближается к сотне\footnote{Запрос к PubMed <<``molecular dynamics simulations'' nucleosome>> выдает 86 записей на момент написания диссератции.}. По мере роста производительности вычислительных ресурсов открываются новые возможности по исследованию динамики нуклеосом на все больших временах. В 2016 году нами были опубликованы результаты по микросекундному моделированию, в данной работе приведены уже результаты по моделированию на временах превышающих 10 микросекунд.

%Оценка термодинамических параметров гидратации малых молекул с высокой точностью путем прямых расчетов методом молекулярной динамики в явном растворителе стала возможной в начале XXI века благодаря прогрессу в скорости вычислений и развитию методов так называемой вычислительной алхимии (расчетов систем с гамильтонианом взаимодействий между частями, зависящим от варьируемого параметра). С тех пор по этой теме было опубликовано достаточно много работ, однако, исследователями рассматривалась гидратация в объеме растворителя, а не вблизи поверхности воды.

%Тема изучения амилоидоподобных фибрилл является весьма популярной тематикой исследований, в том числе методами молекулярного моделирования \cite{lu_understanding_2018}. Благодаря вовлеченности процессов амилоидизации в ряд болезней человека, существенная часть работ посвящена изучению определенных типов фибрилл, в частности фибрилл формируемых амилоидом $\beta$. Для фибрилл, представляющих чрезвычайный медицинский интерес, доступно большое количество экспериментальных данных, в том числе методами ЯМР, РСА \cite{sawaya_atomic_2007}.  Исследования других типов фибрилл являются не настолько активными, для них доступно меньше экспериментальных данных и актуальными являются подходы интегративного моделирования.

%Методы футпринтинга ДНК гидроксильными радикалами развиваются с конца 1980-ых годов \cite{churchill_detection_1990}. Существенный вклад в понимание механизмов расщепления ДНК был внесен группой Т. Туллиуса \cite{balasubramanian_dna_1998}. Этой же группой были заложены основы цифровой обработки данных футпринтинга, реализованной в ряде программ \cite{shadle_quantitative_1997,das_safa_2005}. Однако по состоянию на середину второй декады XXI века программные разработанные ранее программные инструменты не поддерживались разработчиками, и их использование было серьезно затруднено, особенно с целью интеграции получаемых данных в комплексные пайплайны обработки, визуализации и моделирования.

%Интегративное моделирование структуры комплексов нуклеосом с белками хроматина долгое время было затруднительным из-за отсутствия структуры нуклеосомы высокого разрешения, которая была получена в 1997 году \cite{luger_crystal_1997}. Следующим важным шагом, упрощающим построение таких моделей, явились разработки методов описания конформации ДНК в пространстве координат взаимного расположения азотистых оснований с возможностью перехода из огрубленного приближения в атомистическое и оценкой энергии изгиба ДНК \cite{olson_dna_1998,lu_3dna_2003}. На данный момент возможности построения и анализа интегративных моделей нуклеосом с белками хроматина ограничены в основном доступностью качественных экспериментальных данных.
%\fi


%Цели и задачи данной работы
%{\aim} данной работы является развитие подходов построения структурно-динамических моделей биомакромолекулярных комплексов на основе сочетания физического моделирования взаимодействий между атомами с информацией, получаемой различными экспериментальными методами.

{\aim} данной работы является разработка интегративных подходов к построению и методов анализа структурно-динамических моделей больших ДНК-белковых комплексов и амилоидоподобных фибрилл. 



В работе поставлены и решены следующие основные {\tasks}:
\begin{enumerate}

  \item Разработать подходы к интегративному моделированию ДНК-белковых комплексов методом молекулярной динамики и проанализировать динамику нуклеосом с атомистическим уровнем детализации в микросекундном временном диапазоне.


  \item На основе различных экспериментальных данных разработать интегративные подходы к моделированию и верификации моделей комплексов ДНК и белков.
  %Разработать методические основы и программные решения для интегративного моделирования комплексов ДНК и белков на основе различных экспериментальных данных.

  
  %\item Разработать и реализовать подходы для вычисления экспериментально измеримых термодинамических характеристик гидратации и адсорбции малых молекул на основе их атомистических молекулярно-динамических моделей.

  \item Для интегративного моделирования биомакромолекулярных комплексов разработать методы анализа экспериментальных данных по расщеплению ДНК гидроксильными радикалами. Применить данные методы для изучения организации ДНК в нуклеосомах. 

  %Разработать методы анализа и использования экспериментальных данных по расщеплению ДНК гидроксильными радикалами для интегративного моделирования биомакромолекулярных комплексов. Применить данные методы для изучения организации ДНК в нуклеосомах.
  
  \item Разработать подходы и методы построения моделей нуклеосом с белками хроматина. Применить данные подходы для изучения комплексов нуклеосом с белками H1, CENP-C, комплексом белков FACT.

   \item Разработать и применить интегративные подходы по построению атомистических моделей амилоидоподобных фибрилл на основе экспериментальных данных спектроскопии и микроскопии.
\end{enumerate}

{\object}
Предмет исследования состоит в разработке новых системных подходов и алгоритмов построения структурно-динамических моделей биомакромолекулярных комплексов на основе методов молекулярного моделирования с использованием различных экспериментальных данных.
Объектами исследования являлись ДНК-белковые комплексы, нуклеосомы, комплексы нуклеосом с белками хроматина, амилоидоподобные фибриллы, состоящие из пептидов и конъюгатов пептидов с синтетическими полимерами.

%научная новизна
\novelty
%\begin{enumerate}

 Предложен оригинальный интегративный подход к созданию моделей ДНК-белковых комплексов с использованием атомистического моделирования биомакромолекул методами молекулярной динамики, огрубленного молекулярного моделирования ДНК, с учетом свойств симметрии ДНК-белковых комплексов и экспериментальных данных по футпринтингу ДНК, распределению электронной плотности в комплексах, спектроскопических данных о расстояниях между меченными нуклеотидами.


 Впервые на атомистическом уровне в микросекундном временном диапазоне рассчитаны траекторий молекулярной динамики нуклеосом и определены микроконформационные состояния ДНК в процессах ``дыхания'' и отворачивания ДНК от октамера гистонов.  
%\item Впервые предложены методы вычисления термодинамических параметров адсорбции малых молекул на поверхность воды в ходе молекулярно-динамических расчетов, вычислены данные параметры для ряда боковых цепей аминокислот. 

%\item Разработан новый метод, позволяющий установить положение ДНК на нуклеосоме с точностью до одного нуклеотида по данным расщепления ДНК гидроксильными радикалами и установлена структура одной из центромерных нуклеосом пекарских дрожжей.

 Установлена конформация нуклеосом и их комплексов с рядом белков хроматина (белком CENP-C, РНК полимеразами, белками комплекса FACT), установлены структуры ряда амилоидоподобных фибрилл. 

 Методами мультимасштабного молекулярного моделирования с использованием данных ИК-, КД-спектроскопии, рентгеновской дифракции, атомно-силовой микроскопии установлены структуры ряда амилоидоподобных фибрилл.
%Разработаны новые комплексные подходы по установлению и анализу атомистической структуры амилоидоподобных фибрилл, основанные на использовании данных ИК-, КД-спектроскопии, рентгеновской дифракции и методов мультимасштабного атомистического моделирования.

%\end{enumerate}
\ifdefined\DISSER \else \pagebreak \fi
{\influence}

Разработанные в данной работе методы и подходы позволяют решать ряд актуальных научных и практических задач, связанных с изучением структурно-динамических характеристик биомакромолекулярных комплексов. Результаты работы применимы для изучения упаковки ДНК в хроматине на нуклеосомном/супрануклеосомном уровнях и упаковки пептидов в амилоидоподобных фибриллах. 
Структурно-динамические модели нуклеосом и их комплексов могут быть использованы в разработке лекарственных препаратов, направленных на модуляцию эпигенетических механизмов в клетке, например, для поиска ингибиторов связывания белков хроматина с нуклеосомами или ингибиторов посттрансляционных модификаций гистонов. Полученные модели амилоидоподобных фибрилл служат основой для дизайна функциональных самособирающихся филаменов (электропроводящих фибрилл, фибрилл, усиливающих вирусную трансдукцию).

{\methods}
%\ifdefined\DISSER
%Диссертационная работа основана на применении и разработке разнообразных методов компьютерного молекулярного моделирования, в том числе интегративных подходов. Важным методом использованным в работе является суперкомпьютерное моделирование методом атомистической молекулярной динамики в явном растворителе. Использовались программные пакеты NAMD \cite{phillips_scalable_2005}, LAMMPS \cite{noauthor_lammps_nodate} и Gromacs \cite{abraham_gromacs:_2015} различных версий, силовые поля OPLS-AA \cite{jorgensen_development_1998}, AMBER \cite{maier_ff14sb_2015}, CHARMM \cite{best_optimization_2012}, PCFF \cite{maple_derivation_1994} с различными моделями воды и параметрами ионов. Для расчета термодинамических параметров гидратации и адсорбции использовались подходы, основанные на плавном выключении взаимодействий между растворителем и растворенной молекулой с последующей оценкой свободной энергии по методу Беннетта, а также расчет средней силы воздействующей на молекулу при ее фиксированном положении. При расчете морфологий амилоидоподобных фибрилл важным явилось использование термостата на основе метода диссипативной динамики частиц \cite{hoogerbrugge_simulating_1992}. Для задач по численной обработке данных расщепления ДНК гидроксильными радикалами была написана программа HYDROID, реализующая аппроксимацию экспериментальных данных многопараметрической аналитической функцией с помощью алгоритма Левенберга — Марквардта. В задачах интегративного моделирования биомакромолекулярных комплексов на основе данных футпринтинга ДНК применялись алгоритмы расчета доступности атакуемых гидроксильными радикалами атомов по методу Ли и Ричардса \cite{lee_interpretation_1971}, реализованном в программе FreeSASA \cite{mitternacht_freesasa_2016}. При моделировании комплексов нуклеосом с белками хроматина активно использовались огрубленные модели ДНК, основанные на описании геометрии ДНК в виде параметров взаимного расположения пар оснований с заданием гамильтониана в этом пространстве параметров \cite{olson_dna_1998}. 
%Большинство процедур обработки данных, автоматизации и огрубленного моделирования производились с применением языка Python, библиотек NumPy \cite{harris_array_2020}, SciPy \cite{virtanen_scipy_2020}, MDAnalysis \cite{gowers_mdanalysis_2016} и других.
%\else

Разработанные в работе комплексный подход к изучению структуры и динамики биомакромолекулярных комплексов основан на комбинации
%Методология исследования основана на комбинировании 
различных методов атомистической молекулярной динамики, огрубленного моделирования, оптимизации геометрии молекулярной системы при заданных ограничениях на экспериментально определенные параметры системы, отбора конформаций молекулярной системы, соответствующих определенным экспериментальным параметрам, оцифровки экспериментальных данных, с целью определения численных значений параметров, используемых при моделировании.

Методы атомистической молекулярной динамики позволяют проследить за характером временной эволюции, оценить оптимальность стартовой структуры молекулярной системы, изучить ее конформационную релаксацию, охарактеризовать ансамбль динамических состояний стартовой структуры.  В результате МД расчетов становится возможным оценить различные макропараметры системы, связанные с набором динамических состояний системы, и сравнить их значения с экспериментальными данными. В случае нуклеосом таким параметром, например, является доступность нуклеотидов для атаки и расщепления гидроксильными радикалами, которую можно сравнить с результатами экспериментов по гидроксильному футпринтингу. В случае амилоидоподобных фибрилл -- шаг спирали фибриллы, который можно сравнить с данными атомно-силовой микроскопии, а структурные факторы -- с данными рентгеновской дифракции.


Используемые нами методы огрубленного моделирования основаны на представлении ДНК в виде последовательности  Уотсон-Криковских пар оснований, где каждая пара представляется в виде плоского абстрактного элемента. Взаимное расположение соседних вдоль по цепи элементов определяется шестью параметрами Rise, Shift, Slide, Twist, Tilt, Roll (три отвечают за поступательные, а три за вращательные степени свободы). Функция потенциальной энергии задается в гармоническом динуклеотидном приближении, то есть зависит от квадратов смещений этих параметров вдоль цепи ДНК. Такое представление ДНК позволяет достаточно быстро проводить оптимизацию ее геометрии путем поиска минимума энергии. Задание внешних экспериментальных ограничений, например, ограничений на расстояния между определенными нуклеотидами, измеренные методом Ферстеровского резонансного переноса энергии (FRET), позволяет рассчитывать модели, удовлетворяющие экспериментальным данным.

Важным методологическим подходом в нашем исследовании являлся также метод (автоматизированного или полу-автоматизированного) перебора возможной структурной организации биомакромолекулярного комплекса среди большого набора детерминированных вариантов и оценка его соответствия экспериментальным данным, а также оценка его относительной энергетической выгодности. Таким подходом решались, в частности, задачи по поиску типов укладки бета-листов в амилоидоподобных фибриллах. Например, в данных ИК- и КД-спектров проявление пиков на определенной длине волны можно связать с наличием в структуре бета-листов с параллельной или антипараллельной укладкой бета-нитей, данные рентгеновской дифракции с определенным расстоянием между бета-листами. По этим параметрам можно проводить отбор конформаций генерируемых молекулярных систем, соответствующих определенной стартовой структуре и ее эволюции. Аналогично, при построении моделей нуклеосом с определенным положением ДНК, можно использовать данные гидроксильного футпринтинга ДНК для отбора моделей с правильным положением ДНК на нуклеосоме.





%Диссертационная работа основана на применении идей и современных методов компьютерного молекулярного моделирования, в том числе интегративных подходов. Важным методом, использованным в работе, является суперкомпьютерное моделирование методом атомистической молекулярной динамики в явном растворителе. Использовались программные пакеты NAMD, LAMMPS и Gromacs различных версий, силовые поля OPLS-AA, AMBER, CHARMM, PCFF  с различными моделями воды и параметрами ионов.  При расчете морфологий амилоидоподобных фибрилл важным явилось использование термостата на основе метода диссипативной динамики частиц. Для задач по численной обработке данных расщепления ДНК гидроксильными радикалами была написана программа HYDROID, реализующая аппроксимацию экспериментальных данных многопараметрической аналитической функцией с помощью алгоритма Левенберга — Марквардта. В задачах интегративного моделирования биомакромолекулярных комплексов на основе данных футпринтинга ДНК применялись алгоритмы расчета доступности атакуемых гидроксильными радикалами атомов по методу Ли и Ричардса, реализованном в программе FreeSASA. При моделировании комплексов нуклеосом с белками хроматина активно использовались огрубленные модели ДНК, основанные на описании геометрии ДНК в виде параметров взаимного расположения пар оснований с заданием гамильтониана в этом пространстве параметров. 

При моделировании методами молекулярной динамики использовались программные пакеты NAMD, LAMMPS и Gromacs различных версий, силовые поля AMBER, CHARMM, PCFF  с различными моделями воды и параметрами ионов.
Большинство процедур обработки данных, автоматизации и огрубленного моделирования производились с применением языка Python, библиотек NumPy, SciPy, MDAnalysis и других. В ходе работы был разработан ряд программных библиотек для обработки экспериментальных данных (HYDORID, HYDROID\_GUI), а также алгоритмы и конвейеры для моделирования нуклеосом, комплексов нуклеосом с белками хроматина, амилоидоподобных фибрилл.
%\fi

\defpositions


%положения выносимые на защиту
%Положения выносимые на защиту.

%%%%
%%%%Основные постулаты написания положений в диссертации

%Положения могут содержать следующие элементы: 

%авторские или уточненные автором определения
%научные выводы автора
%основополагающие принципы изученной темы
%классификации и характеристики определенных категорий
%перечни
%предложения
%пути совершенствования объекта изучения и т.д.

%Обычно введение включает в себя 3-6 пунктов положений, рядом с номером пункта необходимо написать краткое содержание рассмотренной и решенной автором задачи. Ниже представлены примеры фраз, с которых они начинаются:

%«Разработаны основные научные выводы»;
%«На защиту выносятся следующие результаты научной деятельности…»;
%«На защиту выносятся следующие новые и содержащие элементы новизны основные идеи»;
%«В ходе работы выявлены факторы, которые влияют на…»;
%«Выявлена взаимосвязь между основными элементами…»;
%«Определена целесообразность внедрения…» и т.д.
%Такие выводы должны быть представлены два раза – в автореферате и, непосредственно, во введении. 
%%%%
%\begin{enumerate}
%гл1
%\item Разработанные методы интегративного моделирования позволяют создавать структурно-динамические модели биомакромолекул и их комплексов на основе наборов разнородных экспериментальных данных.

%\item Предложены комплексные подходы по моделированию структуры и динамики ДНК-белковых комплексов с учетом разнородных экспериментальных данных, в том числе низкого информационного содержания.
%гл2

  %\item Показано, что экспериментальные данные по расщеплению ДНК гидроксильными радикалами могут эффективно использоваться для построения структуры комплексов ДНК-белок. Наличие у комплекса оси симметрии второго порядка помогает установить точное (с точностью до одного нуклеотида) положение ДНК относительно белка в комплексе. Такой подход, в частности, позволяет реконструировать структуры нуклеосом с неизвестным положением ДНК на гистоновом октамере.


  %\item Разработанные методы интегративного моделирования позволяют рассчитывать структурно-динамические модели супрануклеосомной структуры хроматина, описывать эффекты связывания белков хроматина с нуклеосомами.


 % \item Методы молекулярной динамики и диссипативной динамики частиц позволяют устанавливать связь между расположением бета-нитей в амилоидоподобных фибриллах и их крупномасштабными геометрическими характеристиками. Комбинация данных подходов с анализом экспериментальных данных (например, данных ИК и КД спектроскопии, атомно-силовой и электронной микроскопии, дифракции рентгеновских лучей на фибриллах) позволяет построить структурные модели амилоидоподобных фибрилл. 

%\end{enumerate}

% Ниже то, что согласовано с рубиным.
% Концептуально обоснованы и реализованы интегративные подходы создания моделей нуклеосом, комплексов нуклеосом с белками хроматина, амилоидоподобных фибрилл на основе методов атомистического моделирования биомакромолекул методами молекулярной динамики, огрубленного молекулярного моделирования и методов учета разнородных экспериментальных данных рентгеноструктурного анализа, атомно-силовой и электронной микроскопии, футпринтинга ДНК, ИК-, КД-спектроскопии, измерений расстояний между флуоресцентными метками на основе эффекта Ферстеровского резонансного переноса энергии.

%   Методами суперкомпьютерной атомистической молекулярной динамики при моделировании в микросекундном временном диапазоне, методами интегративного моделирования воспроизведены функционально важные крупномасштабные конформационные перестройки структуры ДНК-белковых комплексов (отворот ДНК от октамера гистонов, диффузия гистоновых хвостов вдоль ДНК) и построены модели нуклеосом, комплексов нуклеосом с белками хроматина, амилоидоподобных фибрилл.

\begin{enumerate}

\item	Для построения структурно-динамических моделей сложных биомакромолекулярных комплексов концептуально обосновано применение новых интегративных подходов на основе сочетания методов атомистического и огрубленного молекулярного моделирования, методов учета разнородных экспериментальных данных рентгеноструктурного анализа, атомно-силовой и электронной микроскопии, футпринтинга ДНК, ИК-, КД-спектроскопии, измерений расстояний между флуоресцентными метками на основе эффекта Ферстеровского резонансного переноса энергии.
\item	С использованием разработанного интегративного подхода возможно создание атомистических моделей нуклеосом и комплексов нуклеосом с белками хроматина, при этом учет симметрии белковых комплексов позволяет значительно повысить точность построения молекулярных моделей на основе данных футпринтинга ДНК.
\item	Интегративное моделирование позволяет воспроизвести на атомистическом уровне функциональную динамику нуклеосом, важную с точки зрения эпигенетической регуляции функционирования генома, включая крупномасштабные конформационные перестройки структуры ДНК-белковых комплексов (углы входа-выхода ДНК в нуклеосоме, диффузия гистоновых хвостов вдоль ДНК), а также позволяет обнаружить новые моды динамической подвижности, связанные с изменением конформации ДНК, перестройкой взаимодействий гистоновых хвостов, деформацией глобулярных доменов гистонов.
\item	Разработанный подход интегративного моделирования применим для построения молекулярных моделей амилоидоподобных фибрилл, реконструкции укладки пептидов в фибриллах и установления связи между морфологией фибриллы и межмолекулярной укладкой пептидов.
\end{enumerate}



\reliability\ полученных результатов обеспечивается их публикацией в рецензируемых журналах международного уровня с высокими импакт-факторами. Результаты находятся в соответствии с результатами, полученными другими авторами. Материалы диссертационной работы докладывались и обсуждались на научных семинарах в МГУ, ряде международных университетов и исследовательских центров (включая, Национальные Институты Здоровья, США, Университет г. Ульма, Германия, Университет Джонса Хопкинса, США, Классический Университет Центрального Китая, Китай и др.). 
Основные результаты работы докладывались на следующих научных конференциях:
Biophysical Society 64th Annual Meeting  Сан-Диего, США, 15-19 февраля 2020; The 44th FEBS Congress, Краков, Польша, 6-11 июля 2019; Multiscale Modeling of Chromatin: Bridging Experiment with Theory, Les Houches, Франция, 31 марта - 5 апреля 2019; Keystone Symposia on Molecular and Cellular Biology ``Genomic instability and DNA repair'', Санта Фе, США, 2-7 апреля 2017;  Biophysical Society 60th Annual Meeting, Los Angeles, США, 27 февраля - 2 марта 2016;
27 Международная конференция ``Математика. Компьютер. Образование'' г. Дубна, Государственный университет ``Дубна'', Россия, 27 января - 1 февраля 2020; Russian International Conference on Cryo-Electron Microscopy 2019, Москва, Россия, 2-5 июня 2019;
Russian International Conference on Cryo-Electron Microscopy 2019, Москва, Россия, 2-5 июня 2019;
``Ломоносовские чтения - 2019'' Секция ``Биология'', Москва, МГУ, Россия, 15-25 апреля 2019; XXVI международная конференция ``Математика. Компьютер. Образование'', Пущино, Россия, 28 января - 2 февраля 2019 и  др. \ifdefined\DISSER Ряд тезисов опубликованы в научных журналах (см. позиции 38-54 списка публикаций \cite{hada_histone_2019,bass_effect_2019,armeev_linking_2019,shaytan_structural_2018,gorkovets_joint_2018,xiao_molecular_2017,shaytan_hydroxyl-radical_2017,gribkova_investigation_2017,el_kennani_ms_histonedb_2017,chertkov_dual_2017,armeev_modeling_2016,armeev_nucleosome_2016,biswas_genomic_2016,draizen_histonedb_2016,lyubitelev_structure_2016,shaitan_dynamics_2016,shaytan_coupling_2016,shaytan_trajectories_2016,valieva_large-scale_2016,armeev_conformational_2015,armeev_molecular_2015,frank_direct_2015,gaykalova_structural_2015,goncearenco_structural_2015,shaytan_nucleosome_2015,bozdaganyan_comparative_2014,chang_analysis_2014,kasimova_voltage-gated_2014,nishi_physicochemical_2014,sokolova_genome_2014,yolamanova_peptide_2013,shaitan_influence_2013,orekhov_calculation_2012,shaytan_self-assembling_2011,shaytan_self-organizing_2011,%=======
greshnova_sinteticheskaya_2019,armeev_modelirovanie_2013,
          armeev_integrative_2020,kniazeva_analyzing_2020,armeev_analyzing_2019,gribkova_construction_2019,armeev_python_2019,gorkovets_mutual_2018,shaytan_microsecond_2017,shaytan_nucleosome_2016,shaytan_combined_2015,shaytan_polymorphism_2015,kasimova_investigation_2014,kasimova_molecular_2014,chang_pausing_2013,%далее диссертации
          %не вак
          armeev_abstract_2019,bass_abstract_2019,shaytan_water_2014,chang_structural_2013}). \fi
%Студенческий биохимический форум, Биологический факультет МГУ, Москва, Россия, 17 декабря 2018; Moscow Conference on Computational Molecular Biology (MCCMB'17), Москва, Россия, 27-30 июля 2017;
%Российская международная конференция по криоэлектронной микроскопии RICCEM-2017, Москва, МГУ имени М. В. Ломоносова, Россия, 6-8 июня 2017; V International Scientific Conference of Young Researchers Devoted to the 94-th Anniversary of Azerbaijani National Leader Heydar Aliyev, Баку, Азербайджан, 2 апреля 2017 - 19 апреля 2019;
%Первый Российский кристаллографический конгресс, Москва, Россия, 21-26 ноября 2016;
%Gordon Research Conference: Epigenetics: Mechanisms and Implications, 4-9 August 2013, Bryant University, Smithfield, RI; Gordon Research Conference: Chromatin Structure \& Function: Regulation of Chromatin Assembly and Genome Functions, June 8-13, 2014, Bentley University, Waltham, MA; Gordon Research Conference: Human Single Nucleotide Polymorphisms \& Disease: Understanding the Genetic Origin of Human Diseases and Natural Differences, August 3-8, 2014, Stonehill College, Easton, MA; Asilomar Chromatin, Chromosomes and Epigenetics Conference December 10-13, 2015, Pacific Grove CA; VII Московский международный конгресс ``Биотехнология: Состояние и перспективы развития'' 19-22 марта 2013 г, Москва, Россия, 2013.


\contribution\ Основные идеи, методология и результаты исследований, изложенные в диссертации, получены автором лично.
%, либо автор играл существенную роль в постановке задач, их решении, интерпретации результатов и руководстве исследованиями. Данный факт подтверждается тем, что автор является первым, вторым (в том числе с указанием равноценности вклада авторов) или последним соавтором в большом числе публикаций по теме диссертации.
Вклад автора в совместных экспериментально-теоретических работах заключается в применении разработанных методов, моделировании и интерпретации экспериментальных результатов, полученных коллегами.

%\publications\ Основные результаты по теме диссертации изложены в ХХ печатных изданиях~\cite{Sokolov,Gaidaenko,Lermontov,Management},
%Х из которых изданы в журналах, рекомендованных ВАК~\cite{Sokolov,Gaidaenko}, 
%ХХ --- в тезисах докладов~\cite{Lermontov,Management}.


% %%% Реализация пакетом biblatex через движок biber
%  \begin{refsection}[bl-author, bl-registered]
%   % \begin{refsection}[bl-author]

%         % Это refsection=1.
%         % Процитированные здесь работы:
%         %  * подсчитываются, для автоматического составления фразы "Основные результаты ..."
%         %  * попадают в авторскую библиографию, при usefootcite==0 и стиле `\insertbiblioauthor` или `\insertbiblioauthorgrouped`
%         %  * нумеруются там в зависимости от порядка команд `\printbibliography` в этом разделе.
%         %  * при использовании `\insertbiblioauthorgrouped`, порядок команд `\printbibliography` в нём должен быть тем же (см. biblio/biblatex.tex)
%         %
%         % Невидимый библиографический список для подсчёта количества публикаций:
%         \printbibliography[heading=nobibheading, section=1, env=countauthorvak,          keyword=biblioauthorvak]%
%         \printbibliography[heading=nobibheading, section=1, env=countauthorwos,          keyword=biblioauthorwos]%
%         \printbibliography[heading=nobibheading, section=1, env=countauthorscopus,       keyword=biblioauthorscopus]%
%         \printbibliography[heading=nobibheading, section=1, env=countauthorconf,         keyword=biblioauthorconf]%
%         \printbibliography[heading=nobibheading, section=1, env=countauthorother,        keyword=biblioauthorother]%
%          \printbibliography[heading=nobibheading, section=1, env=countregistered,         keyword=biblioregistered]%
%        \printbibliography[heading=nobibheading, section=1, env=countauthorpatent,       keyword=biblioauthorpatent]%
%          \printbibliography[heading=nobibheading, section=1, env=countauthorprogram,      keyword=biblioauthorprogram]%
%         \printbibliography[heading=nobibheading, section=1, env=countauthor,             keyword=biblioauthor]%
%         \printbibliography[heading=nobibheading, section=1, env=countauthorvakscopuswos, filter=vakscopuswos]%
%         \printbibliography[heading=nobibheading, section=1, env=countauthorscopuswos,    filter=scopuswos]%
%         \printbibliography[heading=nobibheading, section=1, env=countauthorarticle,    keyword=article]%
%         %
%         \nocite{*}%
%     \end{refsection}%
%         {\publications} Основные результаты по теме диссертации изложены в~\arabic{citeauthor}~печатных работах \cite{armeev_linking_2019,hada_histone_2019,shaytan_structural_2018,shaytan_hydroxyl-radical_2017,xiao_molecular_2017,draizen_histonedb_2016,el_kennani_ms_histonedb_2017,shaytan_coupling_2016,valieva_large-scale_2016,gaykalova_structural_2015,frank_direct_2015,nishi_physicochemical_2014,yolamanova_peptide_2013,shaytan_self-assembling_2011,shaytan_self-organizing_2011,shaytan_large-scale_2010,shaytan_free_2010,shaytan_solvent_2009,kasimova_voltage-gated_2014,chang_analysis_2014,sokolova_genome_2014,shaytan_nucleosome_2016,chertkov_dual_2017,goncearenco_structural_2015,gorkovets_joint_2018,gribkova_investigation_2017,levtsova_molecular_2009,lyubitelev_structure_2016,nikolaev_dynamics_2010,nikolaev_vzaimodeistvie_2010,orekhov_calculation_2012,shaitan_dynamics_2016,shaitan_influence_2013,shaytan_molekularnaya_2005,shaytan_neravnovesnaya_2006,shaytan_trajectories_2016,%=======
%           armeev_analyzing_2019,armeev_conformational_2015,armeev_integrative_2020,kniazeva_analyzing_2020,armeev_linking_2019,armeev_modeling_2016,armeev_molecular_2015,armeev_nucleosome_2016,armeev_python_2019,bass_effect_2019,biswas_genomic_2016,bozdaganyan_comparative_2014,chang_analysis_2014,chang_pausing_2013,chang_structural_2013,chertkov_dual_2017,draizen_histonedb_2016,el_kennani_ms_histonedb_2017,frank_direct_2015,gaykalova_structural_2015,goncearenco_structural_2015,gorkovets_joint_2018,gorkovets_mutual_2018,gribkova_construction_2019,gribkova_investigation_2017,hada_histone_2019,kasimova_investigation_2014,kasimova_molecular_2014,kasimova_voltage-gated_2014,levtsova_molecular_2009,lyubitelev_structure_2016,nikolaev_dynamics_2010,nikolaev_vzaimodeistvie_2010,nishi_physicochemical_2014,orekhov_calculation_2012,shaitan_dynamics_2016,shaitan_influence_2013,shaytan_combined_2015,shaytan_coupling_2016,shaytan_free_2010,shaytan_hydroxyl-radical_2017,shaytan_large-scale_2010,shaytan_microsecond_2017,shaytan_molekularnaya_2005,shaytan_neravnovesnaya_2006,shaytan_nucleosome_2015,shaytan_nucleosome_2016,shaytan_polymorphism_2015,shaytan_self-assembling_2011,shaytan_self-organizing_2011,shaytan_solvent_2009,shaytan_structural_2018,shaytan_trajectories_2016,sokolova_genome_2014,valieva_large-scale_2016,xiao_molecular_2017,yolamanova_peptide_2013,%далее диссертации
%           shaytan_thesis_kfmn_2010,shaytan_thesis_ulm_2012,%не вак
%           armeev_abstract_2019,armeev_modelirovanie_2013,bass_abstract_2019,greshnova_sinteticheskaya_2019,shaytan_dinamicheskiy_2006,shaytan_peptide_2006,shaytan_supercomputernoe_2009,shaytan_water_2014},
%         \arabic{citeauthorvak} из которых изданы в журналах, рекомендованных ВАК \cite{armeev_linking_2019,hada_histone_2019,shaytan_structural_2018,shaytan_hydroxyl-radical_2017,xiao_molecular_2017,draizen_histonedb_2016,el_kennani_ms_histonedb_2017,shaytan_coupling_2016,valieva_large-scale_2016,gaykalova_structural_2015,frank_direct_2015,nishi_physicochemical_2014,yolamanova_peptide_2013,shaytan_self-assembling_2011,shaytan_self-organizing_2011,shaytan_large-scale_2010,shaytan_free_2010,shaytan_solvent_2009,kasimova_voltage-gated_2014,chang_analysis_2014,sokolova_genome_2014,%=======
%           armeev_analyzing_2019,armeev_conformational_2015,armeev_integrative_2020,kniazeva_analyzing_2020,armeev_linking_2019,armeev_modeling_2016,armeev_molecular_2015,armeev_nucleosome_2016,armeev_python_2019,bass_effect_2019,biswas_genomic_2016,bozdaganyan_comparative_2014,chang_analysis_2014,chang_pausing_2013,chang_structural_2013,chertkov_dual_2017,draizen_histonedb_2016,el_kennani_ms_histonedb_2017,frank_direct_2015,gaykalova_structural_2015,goncearenco_structural_2015,gorkovets_joint_2018,gorkovets_mutual_2018,gribkova_construction_2019,gribkova_investigation_2017,hada_histone_2019,kasimova_investigation_2014,kasimova_molecular_2014,kasimova_voltage-gated_2014,levtsova_molecular_2009,lyubitelev_structure_2016,nikolaev_dynamics_2010,nikolaev_vzaimodeistvie_2010,nishi_physicochemical_2014,orekhov_calculation_2012,shaitan_dynamics_2016,shaitan_influence_2013,shaytan_combined_2015,shaytan_coupling_2016,shaytan_free_2010,shaytan_hydroxyl-radical_2017,shaytan_large-scale_2010,shaytan_microsecond_2017,shaytan_molekularnaya_2005,shaytan_neravnovesnaya_2006,shaytan_nucleosome_2015,shaytan_nucleosome_2016,shaytan_polymorphism_2015,shaytan_self-assembling_2011,shaytan_self-organizing_2011,shaytan_solvent_2009,shaytan_structural_2018,shaytan_trajectories_2016,sokolova_genome_2014,valieva_large-scale_2016,xiao_molecular_2017,yolamanova_peptide_2013}\sloppy%
%         \ifnum \value{citeauthorarticle}>0%
%             , \arabic{citeauthorarticle} составляют статьи\sloppy%
%         \fi%
%         \ifnum \value{citeauthorscopuswos}>0%
%             , \arabic{citeauthorscopuswos} "--- в~периодических научных журналах, индексируемых Web of~Science и Scopus\sloppy%
%         \fi%
%         \ifnum \value{citeauthorconf}>0%
%             , \arabic{citeauthorconf} "--- тезисы докладов и труды конференций.
%         \else%
%             .
%         \fi%
%         \ifnum \value{citeregistered}=1%
%            \ifnum \value{citeauthorpatent}=1%
%                Зарегистрирован \arabic{citeauthorpatent} патент.
%            \fi%
%            \ifnum \value{citeauthorprogram}=1%
%                Зарегистрирована \arabic{citeauthorprogram} программа для ЭВМ.
%            \fi%
%         \fi%
%         \ifnum \value{citeregistered}>1%
%            Зарегистрированы\ %
%            \ifnum \value{citeauthorpatent}>0%
%            \formbytotal{citeauthorpatent}{патент}{}{а}{} \cite{patbib1}\sloppy%
%            \ifnum \value{citeauthorprogram}=0 . \else \ и~\fi%
%            \fi%
%            \ifnum \value{citeauthorprogram}>0%
%            \formbytotal{citeauthorprogram}{программ}{а}{ы}{} для ЭВМ \cite{progbib1}.
%            \fi%
%         \fi%
%         % К публикациям, в которых излагаются основные научные результаты диссертации на соискание учёной
%         % степени, в рецензируемых изданиях приравниваются патенты на изобретения, патенты (свидетельства) на
%         % полезную модель, патенты на промышленный образец, патенты на селекционные достижения, свидетельства
%         % на программу для электронных вычислительных машин, базу данных, топологию интегральных микросхем,
%         % зарегистрированные в установленном порядке.(в ред. Постановления Правительства РФ от 21.04.2016 N 335)
  

%  \begin{refsection}[bl-author, bl-registered]
%  % \begin{refsection}[bl-author]
%         % Это refsection=2.
%         % Процитированные здесь работы:
%         %  * попадают в авторскую библиографию, при usefootcite==0 и стиле `\insertbiblioauthorimportant`.
%         %  * ни на что не влияют в противном случае
%       %  \nocite{vakbib2}%vak
%        %AKShPAT  \nocite{patbib1}%patent
%      %AKShPAT   \nocite{progbib1}%program
%       %  \nocite{bib1}%other
% \nocite{
%         %high impact in right order (do not remove or change)
%         %64 статьи и тезисы + 2 диссертации (тут не учитываются)
%         %VAK изданий 56, не вак 8, тезисов 17, статей 47
%         %NB no of these should appear in othercites.bib (!) - they will then disappear from author list of papers
%        armeev_linking_2019,hada_histone_2019,shaytan_structural_2018,shaytan_hydroxyl-radical_2017,xiao_molecular_2017,draizen_histonedb_2016,el_kennani_ms_histonedb_2017,shaytan_coupling_2016,valieva_large-scale_2016,gaykalova_structural_2015,frank_direct_2015,nishi_physicochemical_2014,yolamanova_peptide_2013,shaytan_self-assembling_2011,shaytan_self-organizing_2011,shaytan_large-scale_2010,shaytan_free_2010,shaytan_solvent_2009,kasimova_voltage-gated_2014,chang_analysis_2014,sokolova_genome_2014,shaytan_nucleosome_2016,chertkov_dual_2017,goncearenco_structural_2015,gorkovets_joint_2018,gribkova_investigation_2017,levtsova_molecular_2009,lyubitelev_structure_2016,nikolaev_dynamics_2010,nikolaev_vzaimodeistvie_2010,orekhov_calculation_2012,shaitan_dynamics_2016,shaitan_influence_2013,shaytan_molekularnaya_2005,shaytan_neravnovesnaya_2006,shaytan_trajectories_2016,%=======
%           armeev_analyzing_2019,armeev_conformational_2015,armeev_integrative_2020,kniazeva_analyzing_2020,armeev_linking_2019,armeev_modeling_2016,armeev_molecular_2015,armeev_nucleosome_2016,armeev_python_2019,bass_effect_2019,biswas_genomic_2016,bozdaganyan_comparative_2014,chang_analysis_2014,chang_pausing_2013,chang_structural_2013,chertkov_dual_2017,draizen_histonedb_2016,el_kennani_ms_histonedb_2017,frank_direct_2015,gaykalova_structural_2015,goncearenco_structural_2015,gorkovets_joint_2018,gorkovets_mutual_2018,gribkova_construction_2019,gribkova_investigation_2017,hada_histone_2019,kasimova_investigation_2014,kasimova_molecular_2014,kasimova_voltage-gated_2014,levtsova_molecular_2009,lyubitelev_structure_2016,nikolaev_dynamics_2010,nikolaev_vzaimodeistvie_2010,nishi_physicochemical_2014,orekhov_calculation_2012,shaitan_dynamics_2016,shaitan_influence_2013,shaytan_combined_2015,shaytan_coupling_2016,shaytan_free_2010,shaytan_hydroxyl-radical_2017,shaytan_large-scale_2010,shaytan_microsecond_2017,shaytan_molekularnaya_2005,shaytan_neravnovesnaya_2006,shaytan_nucleosome_2015,shaytan_nucleosome_2016,shaytan_polymorphism_2015,shaytan_self-assembling_2011,shaytan_self-organizing_2011,shaytan_solvent_2009,shaytan_structural_2018,shaytan_trajectories_2016,sokolova_genome_2014,valieva_large-scale_2016,xiao_molecular_2017,yolamanova_peptide_2013,%далее диссертации
%           shaytan_thesis_kfmn_2010,shaytan_thesis_ulm_2012,%не вак
%           armeev_abstract_2019,armeev_modelirovanie_2013,bass_abstract_2019,greshnova_sinteticheskaya_2019,shaytan_dinamicheskiy_2006,shaytan_peptide_2006,shaytan_supercomputernoe_2009,shaytan_water_2014,%патенты
%           patbib1,progbib1}
%     \end{refsection}%
%         %
%         % Всё, что вне этих двух refsection, это refsection=0,
%         %  * для диссертации - это нормальные ссылки, попадающие в обычную библиографию
%         %  * для автореферата:
%         %     * при usefootcite==0, ссылка корректно сработает только для источника из `external.bib`. Для своих работ --- напечатает "[0]" (и даже Warning не вылезет).
%         %     * при usefootcite==1, ссылка сработает нормально. В авторской библиографии будут только процитированные в refsection=0 работы.




%При использовании пакета \verb!biblatex! для автоматического подсчёта
%количества публикаций автора по теме диссертации, необходимо
%их здесь перечислить с использованием команды \verb!\nocite!.




%% на случай ошибок оставляю исходный кусок на месте, закомментированным
%Полный объём диссертации составляет  \ref*{TotPages}~страницу с~\totalfigures{}~рисунками и~\totaltables{}~таблицами. Список литературы содержит \total{citenum}~наименований.
%

%AKSh - перенес в intioduction
%Полный объём диссертации составляет \formbytotal{TotPages}{страниц}{у}{ы}{} 
%с~\formbytotal{totalcount@figure}{рисунк}{ом}{ами}{ами}
%и~\formbytotal{totalcount@table}{таблиц}{ей}{ами}{ами}. Список литературы содержит  
%\formbytotal{citenum}{наименован}{ие}{ия}{ий}.


%
%
%Во введении к диссертации определяется актуальность избранной темы, степень ее разработанности, цели и задачи, научная новизна, теоретическая и практическая значимость работы, методология диссертационного исследования, положения, выносимые на защиту, степень достоверности и апробация результатов.
 % Характеристика работы по структуре во введении и в автореферате не отличается (ГОСТ Р 7.0.11, пункты 5.3.1 и 9.2.1), потому её загружаем из одного и того же внешнего файла, предварительно задав форму выделения некоторым параметрам





%%% Реализация пакетом biblatex через движок biber
 \begin{refsection}[bl-author, bl-registered]
  % \begin{refsection}[bl-author]

        % Это refsection=1.
        % Процитированные здесь работы:
        %  * подсчитываются, для автоматического составления фразы "Основные результаты ..."
        %  * попадают в авторскую библиографию, при usefootcite==0 и стиле `\insertbiblioauthor` или `\insertbiblioauthorgrouped`
        %  * нумеруются там в зависимости от порядка команд `\printbibliography` в этом разделе.
        %  * при использовании `\insertbiblioauthorgrouped`, порядок команд `\printbibliography` в нём должен быть тем же (см. biblio/biblatex.tex)
        %
        % Невидимый библиографический список для подсчёта количества публикаций:
        \printbibliography[heading=nobibheading, section=1, env=countauthorvak,          keyword=biblioauthorvak]%
        \printbibliography[heading=nobibheading, section=1, env=countauthorwos,          keyword=biblioauthorwos]%
        \printbibliography[heading=nobibheading, section=1, env=countauthorscopus,       keyword=biblioauthorscopus]%
        \printbibliography[heading=nobibheading, section=1, env=countauthorconf,         keyword=biblioauthorconf]%
        \printbibliography[heading=nobibheading, section=1, env=countauthorother,        keyword=biblioauthorother]%
         \printbibliography[heading=nobibheading, section=1, env=countregistered,         keyword=biblioregistered]%
       \printbibliography[heading=nobibheading, section=1, env=countauthorpatent,       keyword=biblioauthorpatent]%
         \printbibliography[heading=nobibheading, section=1, env=countauthorprogram,      keyword=biblioauthorprogram]%
        \printbibliography[heading=nobibheading, section=1, env=countauthor,             keyword=biblioauthor]%
        \printbibliography[heading=nobibheading, section=1, env=countauthorvakscopuswos, filter=vakscopuswos]%
        \printbibliography[heading=nobibheading, section=1, env=countauthorscopuswos,    filter=scopuswos]%
        \printbibliography[heading=nobibheading, section=1, env=countauthorarticle,    keyword=article]%
        %
        \nocite{*}%
    \end{refsection}%
        {\publications} Основные результаты по теме диссертации изложены в~\arabic{citeauthor}~печатных работах \cite{hada_histone_2019,bass_effect_2019,armeev_linking_2019,shaytan_structural_2018,gorkovets_joint_2018,xiao_molecular_2017,shaytan_hydroxyl-radical_2017,gribkova_investigation_2017,el_kennani_ms_histonedb_2017,chertkov_dual_2017,armeev_modeling_2016,armeev_nucleosome_2016,biswas_genomic_2016,draizen_histonedb_2016,lyubitelev_structure_2016,shaitan_dynamics_2016,shaytan_coupling_2016,shaytan_trajectories_2016,valieva_large-scale_2016,armeev_conformational_2015,armeev_molecular_2015,frank_direct_2015,gaykalova_structural_2015,goncearenco_structural_2015,shaytan_nucleosome_2015,bozdaganyan_comparative_2014,chang_analysis_2014,kasimova_voltage-gated_2014,nishi_physicochemical_2014,sokolova_genome_2014,yolamanova_peptide_2013,shaitan_influence_2013,orekhov_calculation_2012,shaytan_self-assembling_2011,shaytan_self-organizing_2011,%=======
          armeev_integrative_2020,kniazeva_analyzing_2020,armeev_analyzing_2019,gribkova_construction_2019,armeev_python_2019,gorkovets_mutual_2018,shaytan_microsecond_2017,shaytan_nucleosome_2016,shaytan_combined_2015,shaytan_polymorphism_2015,kasimova_investigation_2014,kasimova_molecular_2014,chang_pausing_2013,%далее диссертации
          %не вак
          armeev_abstract_2019,bass_abstract_2019,greshnova_sinteticheskaya_2019,shaytan_water_2014,chang_structural_2013,armeev_modelirovanie_2013},
        в том числе в 35 статьях в рецензируемых научных изданиях, индексируемых в базах данных Web of Science, Scopus, RSCI \cite{hada_histone_2019,bass_effect_2019,armeev_linking_2019,shaytan_structural_2018,gorkovets_joint_2018,xiao_molecular_2017,shaytan_hydroxyl-radical_2017,gribkova_investigation_2017,el_kennani_ms_histonedb_2017,chertkov_dual_2017,armeev_modeling_2016,armeev_nucleosome_2016,biswas_genomic_2016,draizen_histonedb_2016,lyubitelev_structure_2016,shaitan_dynamics_2016,shaytan_coupling_2016,shaytan_trajectories_2016,valieva_large-scale_2016,armeev_conformational_2015,armeev_molecular_2015,frank_direct_2015,gaykalova_structural_2015,goncearenco_structural_2015,shaytan_nucleosome_2015,bozdaganyan_comparative_2014,chang_analysis_2014,kasimova_voltage-gated_2014,nishi_physicochemical_2014,sokolova_genome_2014,yolamanova_peptide_2013,shaitan_influence_2013,orekhov_calculation_2012,shaytan_self-assembling_2011,shaytan_self-organizing_2011}\sloppy%
        \ifnum \value{citeauthorarticle}>0%
            , \arabic{citeauthorarticle} составляют статьи\sloppy%
        \fi%
        \ifnum \value{citeauthorscopuswos}>0%
            , \arabic{citeauthorscopuswos} "--- в~периодических научных журналах, индексируемых Web of~Science и Scopus\sloppy%
        \fi%
        \ifnum \value{citeauthorconf}>0%
            , \arabic{citeauthorconf} "--- тезисы докладов и труды конференций.
        \else%
            .
        \fi%
        \ifnum \value{citeregistered}=1%
           \ifnum \value{citeauthorpatent}=1%
               Зарегистрирован \arabic{citeauthorpatent} патент.
           \fi%
           \ifnum \value{citeauthorprogram}=1%
               Зарегистрирована \arabic{citeauthorprogram} программа для ЭВМ.
           \fi%
        \fi%
        \ifnum \value{citeregistered}>1%
           Зарегистрированы\ %
           \ifnum \value{citeauthorpatent}>0%
           \formbytotal{citeauthorpatent}{патент}{}{а}{} \cite{patbib1}\sloppy%
           \ifnum \value{citeauthorprogram}=0 . \else \ и~\fi%
           \fi%
           \ifnum \value{citeauthorprogram}>0%
           \formbytotal{citeauthorprogram}{программ}{а}{ы}{} для ЭВМ \cite{progbib1}.
           \fi%
        \fi%
        % К публикациям, в которых излагаются основные научные результаты диссертации на соискание учёной
        % степени, в рецензируемых изданиях приравниваются патенты на изобретения, патенты (свидетельства) на
        % полезную модель, патенты на промышленный образец, патенты на селекционные достижения, свидетельства
        % на программу для электронных вычислительных машин, базу данных, топологию интегральных микросхем,
        % зарегистрированные в установленном порядке.(в ред. Постановления Правительства РФ от 21.04.2016 N 335)
  

 \begin{refsection}[bl-author, bl-registered]
 % \begin{refsection}[bl-author]
        % Это refsection=2.
        % Процитированные здесь работы:
        %  * попадают в авторскую библиографию, при usefootcite==0 и стиле `\insertbiblioauthorimportant`.
        %  * ни на что не влияют в противном случае
      %  \nocite{vakbib2}%vak
       %AKShPAT  \nocite{patbib1}%patent
     %AKShPAT   \nocite{progbib1}%program
      %  \nocite{bib1}%other
\nocite{
        %high impact in right order (do not remove or change)
        %64 статьи и тезисы + 2 диссертации (тут не учитываются)
        %VAK изданий 56, не вак 8, тезисов 17, статей 47
        %NB no of these should appear in othercites.bib (!) - they will then disappear from author list of papers
hada_histone_2019,bass_effect_2019,armeev_linking_2019,shaytan_structural_2018,gorkovets_joint_2018,xiao_molecular_2017,shaytan_hydroxyl-radical_2017,gribkova_investigation_2017,el_kennani_ms_histonedb_2017,chertkov_dual_2017,armeev_modeling_2016,armeev_nucleosome_2016,biswas_genomic_2016,draizen_histonedb_2016,lyubitelev_structure_2016,shaitan_dynamics_2016,shaytan_coupling_2016,shaytan_trajectories_2016,valieva_large-scale_2016,armeev_conformational_2015,armeev_molecular_2015,frank_direct_2015,gaykalova_structural_2015,goncearenco_structural_2015,shaytan_nucleosome_2015,bozdaganyan_comparative_2014,chang_analysis_2014,kasimova_voltage-gated_2014,nishi_physicochemical_2014,sokolova_genome_2014,yolamanova_peptide_2013,shaitan_influence_2013,orekhov_calculation_2012,shaytan_self-assembling_2011,shaytan_self-organizing_2011,%=======
         armeev_integrative_2020,kniazeva_analyzing_2020,armeev_analyzing_2019,gribkova_construction_2019,armeev_python_2019,gorkovets_mutual_2018,shaytan_microsecond_2017,shaytan_nucleosome_2016,shaytan_combined_2015,shaytan_polymorphism_2015,kasimova_investigation_2014,kasimova_molecular_2014,chang_pausing_2013,%далее диссертации
          %не вак
          armeev_abstract_2019,bass_abstract_2019,greshnova_sinteticheskaya_2019,shaytan_water_2014,chang_structural_2013,armeev_modelirovanie_2013,%патенты
          patbib1,progbib1}
    \end{refsection}%
        %
        % Всё, что вне этих двух refsection, это refsection=0,
        %  * для диссертации - это нормальные ссылки, попадающие в обычную библиографию
        %  * для автореферата:
        %     * при usefootcite==0, ссылка корректно сработает только для источника из `external.bib`. Для своих работ --- напечатает "[0]" (и даже Warning не вылезет).
        %     * при usefootcite==1, ссылка сработает нормально. В авторской библиографии будут только процитированные в refsection=0 работы.










\underline{\textbf{Объем и структура работы.}} Диссертация состоит из~введения, пяти глав и заключения.
Первая глава посвящена обзору методов исследований, примененных и разработанных в ходе работы, обсуждается место методов молекулярного и интегративного моделирования в исследовательском процессе. Вторая глава посвящена применению методов суперкомпьютерной молекулярной динамики для изучения динамики нуклеосом.  Третья глава посвящена разработке методов интегративного моделирования комплексов белков с ДНК на основе данных расщепления ДНК гидроксильными радикалами. Четвертая глава посвящена построению моделей комплексов нуклеосом с белками хроматина. Пятая глава посвящена моделированию амилоидоподобных фибрилл.
%% на случай ошибок оставляю исходный кусок на месте, закомментированным
%Полный объём диссертации составляет  \ref*{TotPages}~страницу
%с~\totalfigures{}~рисунками и~\totaltables{}~таблицами. Список литературы
%содержит \total{citenum}~наименований.
%
Полный объём диссертации составляет
\formbytotal{TotPages}{страниц}{у}{ы}{}, включая
\formbytotal{totalcount@figure}{рисун}{ок}{ка}{ков} и
\formbytotal{totalcount@table}{таблиц}{у}{ы}{}.
Список литературы содержит
\formbytotal{citeexternal}{наименован}{ие}{ия}{ий}.

%AKSh весь блок ниже вставлен, чтобы давать в конце введения список работ.
%Но есто проблема - гиперссылки на них не работают правильно.
%ПОЭТОМУ БЛОК НИЖЕ ЗАКОММЕНТИРОВАН - и дается единый список литературы в конце.
\subsubsection*{}
\textbf{\large Публикации автора по теме диссертации}\footnote{В скобках приведен объем публикации в печатных листах и вклад автора в печатных листах.}
\addcontentsline{toc}{section}{Публикации автора по теме диссертации} 
\ifnumequal{\value{bibliosel}}{0}{% Встроенная реализация с загрузкой файла через движок bibtex8
  \renewcommand{\bibname}{\large \bibtitleauthor}
  \nocite{*}
  \insertbiblioauthor           % Подключаем Bib-базы
  %\insertbiblioexternal   % !!! bibtex не умеет работать с несколькими библиографиями !!!
}{% Реализация пакетом biblatex через движок biber
  % Цитирования.
  %  * Порядок перечисления определяет порядок в библиографии (только внутри подраздела, если `\insertbiblioauthorgrouped`).
  %  * Если не соблюдать порядок "как для \printbibliography", нумерация в `\insertbiblioauthor` будет кривой.
  %  * Если цитировать каждый источник отдельной командой --- найти некоторые ошибки будет проще.
  %
  \nocite{
        %high impact in right order (do not remove or change)
        %64 статьи и тезисы + 2 диссертации (тут не учитываются)
        %VAK изданий 56, не вак 8, тезисов 17, статей 47
        %NB no of these should appear in othercites.bib (!) - they will then disappear from author list of papers
hada_histone_2019,bass_effect_2019,armeev_linking_2019,shaytan_structural_2018,gorkovets_joint_2018,xiao_molecular_2017,shaytan_hydroxyl-radical_2017,gribkova_investigation_2017,el_kennani_ms_histonedb_2017,chertkov_dual_2017,armeev_modeling_2016,armeev_nucleosome_2016,biswas_genomic_2016,draizen_histonedb_2016,lyubitelev_structure_2016,shaitan_dynamics_2016,shaytan_coupling_2016,shaytan_trajectories_2016,valieva_large-scale_2016,armeev_conformational_2015,armeev_molecular_2015,frank_direct_2015,gaykalova_structural_2015,goncearenco_structural_2015,shaytan_nucleosome_2015,bozdaganyan_comparative_2014,chang_analysis_2014,kasimova_voltage-gated_2014,nishi_physicochemical_2014,sokolova_genome_2014,yolamanova_peptide_2013,shaitan_influence_2013,orekhov_calculation_2012,shaytan_self-assembling_2011,shaytan_self-organizing_2011,%=======
         armeev_integrative_2020,kniazeva_analyzing_2020,armeev_analyzing_2019,gribkova_construction_2019,armeev_python_2019,gorkovets_mutual_2018,shaytan_microsecond_2017,shaytan_nucleosome_2016,shaytan_combined_2015,shaytan_polymorphism_2015,kasimova_investigation_2014,kasimova_molecular_2014,chang_pausing_2013,%далее диссертации
          %не вак
          armeev_abstract_2019,bass_abstract_2019,greshnova_sinteticheskaya_2019,shaytan_water_2014,chang_structural_2013,armeev_modelirovanie_2013,%патенты
          patbib1,progbib1}
  %% authorvak
  %\nocite{vakbib1}%
 % \nocite{vakbib2}%
  %
  %% authorwos
 % \nocite{wosbib1}%
  %
  %% authorscopus
 % \nocite{scbib1}%
  %
  %% authorpathent
  %\nocite{patbib1}%
  %
  %% authorprogram
  %\nocite{progbib1}%
  %
  %% authorconf
  %\nocite{confbib1}%
 % \nocite{confbib2}%
  %
  %% authorother
  %\nocite{bib1}%
  %\nocite{bib2}%
    \begin{refcontext}[labelprefix=A]
        \insertbiblioauthor      % Вывод всех работ автора
    \end{refcontext}
  
  %  \insertbiblioauthorimportant  % Вывод наиболее значимых работ автора (определяется в файле characteristic во второй section)

  % Невидимый библиографический список для подсчёта количества внешних публикаций
  % Используется, чтобы убрать приставку "А" у работ автора, если в автореферате нет
  % цитирований внешних источников.
 % \printbibliography[heading=nobibheading, section=0, env=countexternal, keyword=biblioexternal, resetnumbers=true]%
  
}



% \ifthenelse{\equal{\thebibliosel}{0}}{% Встроенная реализация с загрузкой файла через движок bibtex8
%   \renewcommand{\refname}{\large \authorbibtitle}
%   \nocite{*}
%   \insertbiblioauthor                          % Подключаем Bib-базы
%   %\insertbiblioother   % !!! bibtex не умеет работать с несколькими библиографиями !!!
% }{% Реализация пакетом biblatex через движок biber
%   \insertbiblioauthor                          % Подключаем Bib-базы
%  % \insertbiblioother
% }
