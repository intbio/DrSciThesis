\chapter*{Заключение}						% Заголовок
\addcontentsline{toc}{chapter}{Заключение}	% Добавляем его в оглавление

%% Согласно ГОСТ Р 7.0.11-2011:
%% 5.3.3 В заключении диссертации излагают итоги выполненного исследования, рекомендации, перспективы дальнейшей разработки темы.
%% 9.2.3 В заключении автореферата диссертации излагают итоги данного исследования, рекомендации и перспективы дальнейшей разработки темы.
%% Поэтому имеет смысл сделать эту часть общей и загрузить из одного файла в автореферат и в диссертацию:


%\section*{Основные результаты и выводы диссретационной работы}
%\addcontentsline{toc}{section}{Основные результаты и выводы диссретационной работы}
%% Согласно ГОСТ Р 7.0.11-2011:
%% 5.3.3 В заключении диссертации излагают итоги выполненного исследования, рекомендации, перспективы дальнейшей разработки темы.
%% 9.2.3 В заключении автореферата диссертации излагают итоги данного исследования, рекомендации и перспективы дальнейшей разработки темы.


\ifdefined\DISSER  
\section*{Обсуждение результатов}
\addcontentsline{toc}{section}{Обсуждение результатов} 
\else 
\underline{\textbf{Обсуждение результатов}}
\pdfbookmark{Обсуждение результатов}{disc} 
\fi

Построение и анализ структурных моделей биомакромолекулярных комплексов имеет ключевое значение в изучении структурной организации биологических систем. В реализации этого подхода принципиальную роль играют методы компьютерного молекулярного моделирования. %Во-первых, это связано с тем, что любые экспериментальные данные о взаимном расположении молекул априори являются косвенными и требуют вычислительной интерпретации. Во-вторых, методы компьютерного молекулярного моделирования позволяют при построении и анализе моделей использовать знания и фундаментальные представления о взаимодействии атомов и молекул между собой, что может существенно дополнить экспериментальную информацию.
 В данной работе рассмотрены возможности методов молекулярного моделирования в построении и анализе больших ДНК-белковых комплексов и амилодоподобных фибрилл. В результате были разработаны новые интегративные методы моделирования для больших комплексов, для изучения свойств которых недостаточно стандартных.

 Одним из наиболее широкораспространенных и развитых методов молекулярного моделирования является метод молекулярной динамики (МД). Физические модели молекул в методе МД основаны на рассмотрении взаимодействий между атомами в приближении классической механики Ньютона. Согласно этому для биомолекул по некоторым правилам задается эмпирическая функция зависимости потенциальной энергии системы от координат атомов (так называемое силовое поле). Теоретически, при идеальной точности задания силового поля и неограниченных вычислительных возможностях метод МД может работать в режиме ``вычислительного микроскопа'', который воспроизводит реальные процессы структурообразования и взаимодействия биологических молекул. Однако на практике мы имеем дело лишь с некоторым приближенным описанием взаимодействий, построенным по аддитивному принципу. Кроме того мы ограничены во временах моделирования, что делает возможным исследование молекулярных систем лишь в ограниченной области конформационного пространства вблизи их стартовых конформаций. Современное развитие методов МД обычно представляет собой повторяющийся двухстадийный процесс, в котором вслед за совершенствованием вычислительных возможностей следует совершенствование силовых полей. Характерным примером здесь является совершенствование моделей ДНК. Так, силовые поля класса AMBER были изначально параметризованы для воспроизведения экспериментальной геометрии ДНК в наносекундном диапазоне, однако, в середине 2000-ых, оказалось, что при временах моделировании свыше 10 нс изменения торсионных углов остова ДНК приводят к неправильной геометрии ДНК, и поэтому силовое поле было усовершенствовано. Были изменены параметры силового поля, отвечающие за энергетические термы вращения торсионных углов в ДНК, что привело к сохранению правильной В-формы ДНК на доступных временах моделирования. В середине 2010-ых ситуация повторилась, когда стандартные времена моделирования увеличились до микросекундного масштаба, и силовое поле было вновь усовершенствовано. Аналогичные примеры можно привести и для силовых полей белков, воды и ионов. Кроме правильного воспроизведения равновесных параметров молекулярных систем, отдельным вопросом является качество воспроизведения их динамики. Здесь с точки зрения силового поля важным является воспроизведение правильного энергетического баланса (заселенности) между различными конформациями. Мы показали \cite{shaytan_free_2010}, что на больших временах моделирования начинают проявляться энтропийные эффекты, которые в случае белков приводят к излишней стабилизации структурных моделей. На практике при исследовании динамических процессов в биомакромолекулярных комплексах методами МД моделирования, необходимо оценить с одной стороны, в каких временных диапазонах МД модель адекватно описывает систему, а с другой стороны, в каких временных диапазонах возможно компьютерное воспроизведение функционально интересных конформационных перестроек. Результаты нашего моделирования на примере нуклеосом показывают, что современные методы МД позволяют создавать структурно-динамические модели ДНК-белковых комплексов без как-либо видимых артефактов по крайней мере на временах в десятки микросекунд. Свободная ДНК сохраняет конформацию, свойственную В-форме ДНК, ионное окружение является стабильным, глобулярные домены белка являются стабильными, а неупорядоченные домены сохраняют повышенную конформационную подвижность. В тоже время проведенный нами анализ динамики показал, что на временном диапазоне в пределах одной микросекунды наблюдаются такие процессы, как флуктуации формы свободных участков ДНК, конденсация неупорядоченных белковых доменов по поверхность ДНК, появления локальных конформационных возмущений в участках ДНК, связанных с белком. На временных масштабах порядка десятков микросекунд можно уже изучать диссоциацию/реассоциацию ДНК от белка в отдельных сайтах связывания, частичную диффузию неупорядоченных доменов вдоль поверхности ДНК. В то же время, более крупномасштабные перестройки ДНК-белковых комплексов пока находятся за пределами возможностей МД. Так, вероятно, процессы комплексообразования, например, связывания транскрипционных факторов и поиск ими сайтов на последовательности ДНК или процессы образования гетероструктур ДНК-РНК (например, при связывании комплексов системы CRISPR-Cas9) находятся за пределами возможностей МД. Аналогичная ситуация возникает и при моделировании самосборки амилоидоподобных фибрилл. Несмотря на высокую точность задания силовых полей белков по сравнению с силовыми полями нуклеиновых кислот, ситуация здесь усложняется повышенными требованиями к точности определения энергии для различных конформаций пептидов. Известно, что амилоидоподобные фибриллы обладают конформационным полиморфизмом и в экспериментальных работах небольшое изменение условий (состав растворителя, температура) и, соответственно, энергетического баланса может приводить к формированию фибрилл различного типа. 

Именно для решения такого рода задач, мы обратились к развитию и разработке методов интегративного моделирования. Базовая идея разработанных методов интегративного моделирования состоит в объединении информации заложенной в силовых полях с информацией о взаимном расположении элементов структуры, которую мы получаем из дополнительных экспериментальных методов.
Общая логика построения интегративных подходов может быть сформулирована следующим образом. При моделировании структурных элементов и параметров системы, на структуру и значения которых  может сильно повлиять неточность силовых полей, необходимо дополнительно привлекать экспериментальные данные. Вместе с тем, моделирование конформационно устойчивых структурных элементов можно производить на основании данных силовых полей. Примером служат полученные результаты моделирования комплексов нуклеосом с РНК полимеразой или моделирования геометрии линкерной ДНК при связывании гистона H1. Предполагалось, что свободная ДНК в целом должна находится в В-конформации двойной спирали, согласно параметрам, заложенным в силовом поле, описывающем структуру ДНК. В то же время более тонкие параметры, такие как небольшой изгиб ДНК, уровень ее отворота от нуклеосомы определяются ограничениями, накладываемыми согласно дополнительными экспериментальными данными. 
В целом аналогичной логике мы следовали при интегративном моделировании амилоидоподобных фибрилл. Были выделены параметры, которые сложно воспроизвести только с помощью моделирования, а именно, параметры укладки пептидов внутри фибриллы, и параметры, которые воспроизводятся моделированием с хорошей точностью, а именно, связь морфологии фибриллы с межмолекулярной укладкой пептидов. Таким образом, экспериментальные данные использовались частично при выборе укладок, а также для отбора фибрилл на уровне их крупномасштабной морфологии.

Таким образом общий принцип разработки интегративных подходов различных биомакромолекулярных комплексов, который можно рекомендовать, состоит в следующем. В первую очередь с учетом известных экспериментальных и литературных данных проводится анализ различных уровней структурной организации моделируемой системы, выделяются уровни организации, моделирование которых эффективно может быть проведено на основе имеющихся силовых полей, а также уровни организации, при моделировании которых необходимо привлекать дополнительные экспериментальные данные. На следующем этапе необходимо использовать различные интегративные методы моделирования, которые позволяют одновременно учитывать как различные экспериментальные данные, так и данные основанные на представлениях о внутримолекулярной структуре и межмолекулярных взаимодействиях. 

В интегративном моделировании весьма полезным походом является использования соображений структурной симметрии. В некоторых случаях они позволяют кардинальным образом упростить задачу, повысить точность моделирования и уменьшить влияние возможных систематических ошибок в экспериментальных данных. Именно этот подход был продемонстрирован при использовании данных футпринтинга ДНК в построении моделей нуклеосом. На качество профиля футпринтинга вдоль нити ДНК могут оказывать влияние такие факторы, как наличие минорных альтернативных продуктов химического расщепления, локальная динамика ДНК в процессе реакции, шумовой фон. Однако, в силу наличия оси симметрии второго порядка в нуклеосоме такие эффекты будут симметричны для двух цепей ДНК и не влияют на алгоритм поиска положения в ДНК, относительно которого профили симметричны. Аналогичные соображения можно рекомендовать использовать при интегративном моделировании других ДНК белковых систем, обладающих симметрией. К таким системам относятся многие транскрипционные факторы, формирующие гомодимеры. 


\underline{\textbf{Заключение}}
\ifdefined\DISSER   \else \pdfbookmark{Заключение}{conclusion} \fi

В настоящей работе автором развиты новые системные подходы для построения структурно-динамических моделей сложных биомакромолекулярных комплексов. Разработанные интегративные методы молекулярного моделирования основаны на сочетании (интеграции) физических моделей взаимодействия молекул с информацией получаемой из различных источников экспериментальных данных. В ходе исследования развиты подходы, (i) позволяющие проводить анализ динамики атомистических структур в микросекундном временном диапазоне, (ii) учитывать различные экспериментальные данные при построении моделей (например, данные экспериментов по футпринтингу ДНК, данные электронной микроскопии, данные FRET-микроскопии, данные рентгеновской дифракции на фибриллах, атомно-силовой микроскопии и т.д.), (iii) сочетать атомистическое и огрубленное представление молекулярных систем (в частности при моделировании ДНК-содержащих комплексов).


Разработанные автором подходы и методы рекомендуется применять для установления структуры и динамической подвижности биомакромолекулярных комплексов, размер и свойства которых затрудняют применение стандартных методов структурной биологии. Практическое применение данных подходов продемонстрировано в диссертации на примере ряда биомакромолекулярных систем, в том числе нуклеосом, комплексов нуклеосом с белками хроматина, амилоидоподобных фибрилл.



\ifdefined\DISSER  
\section*{Выводы диссертационной работы}
\addcontentsline{toc}{section}{Выводы диссертационной работы} 
\else 
\underline{\textbf{Выводы диссертационной работы}}
\pdfbookmark{Выводы диссертационной работы}{vivodi}
\fi

\begin{enumerate}
  %\item Разработаны подходы и программные решения для мультимасштабного моделирования комплексов ДНК и белков с использованием информации из экспериментов по ДНК футпринтингу, электронной микроскопии, FRET-микроскопии. Комбинированное представление ДНК как в атомистическом, так и в динуклеотидном приближении позволяет проводить быструю крупномасштабную оптимизацию структур и при этом учитывать атомистические детали взаимодействия ДНК с белками.
\item Разработан интегративный подход и программные решения для мультимасштабного моделирования комплексов ДНК и белков, в котором используются различные данные экспериментов по ДНК футпринтингу, электронной микроскопии, FRET-микроскопии и комбинированное представление ДНК в атомистическом и в динуклеотидном приближении.

  % \item Разработанные подходы суперкомпьютерного моделирования нуклеосом позволили изучить динамические характеристики нуклеосом в микросекундном временном диапазоне. Охарактеризованы крупномасштабные функциональные конформационные изменения - флуктуации линкерной  ДНК, диссоциация ДНК от гистонового октамера, образование дефектов кручения ДНК, переключения конформаций гистоновых хвостов.

  \item В микросекундном временном диапазоне определены параметры крупномасштабных функциональных конформационных изменений структуры нуклеосом: вычислен ансамбль конформаций линкерной ДНК, установлено влияние электростатического отталкивания на геометрию сегментов линкерной ДНК, определено характерное время диссоциации концов нуклеосомальной ДНК от гистонового октамера (10 мкс), установлены флуктуационные структурные механизмы образования дефектов кручения ДНК, переключения конформаций гистоновых хвостов.

  %\item Cовременные методы суперкомпьютерной атомистической молекулярной динамики позволяют рассчитывать для малых молекул на основе заданной модели атом-атомных взаимодействий термодинамические параметры гидратации (свободную энергию гидратации и адсорбции) с высокой статистической точностью (ошибка 1 kT и меньше в зависимости от длины расчета). Расчет профилей свободной энергии боковых цепей аминокислот вблизи поверхности воды указывает на наличие минимума свободной энергии на поверхности воды (вблизи участка, где плотность равна половине от объемной плотности). 

  

  \item Обработка экспериментальных данных по расщеплению ДНК гидроксильными радикалами (футпринтинга ДНК) позволила вычислить вероятность расщепления ДНК для каждого нуклеотида. Показано, что профили расщепления ДНК гидроксильными радикалами в нуклеосоме мало зависят от последовательности ДНК, и определяются в основном позиционированием ДНК на нуклеосоме. Предложен алгоритм точного определения положения ДНК в нуклеосоме по данным футпринтинга для двух цепей ДНК на основании положения оси псевдосимметрии.

  \item С помощью методов интегративного моделирования были установлены структуры и параметры взаимодействий комплексов нуклеосом с РНК полимеразами, белком CENP-C, гистоном H1, белками комплекса FACT. Установлено, что в положении +49 после входа в нуклеосому РНК полимераза II может формировать компактный комплекс с нуклеосомой, в котором контакты гистонов с ДНК сохраняются по обе стороны активного центра. В модели центромерной нуклеосомы дрожжей определено положение ДНК и установлено, что белок CENP-C взаимодействует с нуклеосомой в районе 20 нуклеотидов от центра симметрии нуклеосомы. Предложены модели конформации линкерных сегментов ДНК при связывании гистона H1.  Установлены амплитуды конформационной подвижности ДНК в нуклеосомах при связывании с комплексом FACT.
  
  %модели указывают на то, что конформационно-динамический полиморфизм нуклеосом является важным фактором при их взаимодействии с белками хроматина. Структура нуклеосом может серьезным образом изменяется при взаимодействиях с белками хроматина (например, в изученном нами взаимодействии с комплексом FACT). Структура комплексов нуклеосом с белками хроматина также зачастую зависит от деталей их состава и внешних условий (например, продемонстрировано нами при моделировании полиморфизма связывания нуклеосомы с гистоном H1). Структура и ориентация ДНК на нуклеосоме является важным фактором для узнавания белками хроматина, что продемонстрировано в работах по изучению комплексов CENP-C с нуклеосомой.

 \item Разработанные подходы построения моделей амилоидоподобных фибрилл позволили изучить связь взаимного расположения пептидов в фибриллярных структурах с крупномасштабной морфологией амилоидоподобных фибрилл и таким образом установить структурную организацию филаментов на основе диблок олигомеров кватертиофена и пептида $(Thr-Val)_3$, а также на основе фрагмента белка gp120. Установлено, что рассмотренные амилоидоподобные фибриллы могут образовываться на основе двух взаимодействующих бета-листов, которые формируют либо плоскую ленту, либо левозакрученную ленту с периодом 24-30 нм.

\end{enumerate}



\section*{Перспективы}
\addcontentsline{toc}{section}{Перспективы}
Начиная со второй половины двадцатого века и до наших дней, благодаря развитию методов цитологии, молекулярной биологии, структурной биологии достигнут существенный прогресс в понимании устройства живых систем. Мы достаточно хорошо понимаем работу систем на молекулярном уровне - уровне генетического кода ДНК, уровне работы отдельных ферментов. Также, достаточно подробно охарактеризована структура и морфология многих клеток, механизмы различных внутриклеточных процессов и структурных преобразований. Однако, промежуточный супрамолекулярный уровень устройства живых организмов в диапазоне от 10 нм и до пределов разрешения оптической микроскопии остается слабо понятым, как с точки зрения структуры, так и с точки зрения механизмов функционирования.
Причины данных затруднений двояки. С одной стороны биомакромолекулярные комплексы часто обладают динамической структурой, что снижает вероятность их кристаллизации для изучения структуры с помощью методов рентгеновской кристаллографии, усложняет возможности реконструкции методами криоэлектронной микроскопии. В том числе большой размер комплексов препятствует их эффективному изучению методами биомолекулярного ЯМР. С другой стороны на больших пространственных масштабах, в том числи из-за подвижности структуры биомакромолекулярных комплексов, множества различных взаимодействий межу ними, становится сложным проследить связь между структурой и механизмами работы биомакромолекул. Иными словами, если для небольших белков изучение их структуры позволяет понять механизмы их работы, то для больших биомакромолекулярных комплексов, даже имея набор их структур в различных состояниях, сложнее выяснить и понять, какими силами и внешними факторами модулируется переход между различными состояниями и каким образом данный комплекс функционирует в контексте более сложной биологической системы.
Перспективы понимания функционирования живых систем на супрамолекулярном уровне на наш взгляд связаны с совместным развитием нескольких областей: 1) методов структурной биологии, способных генерировать и обрабатывать большие массивы экспериментальных данных для получения информации о конформационных структурных ансамблях биомакромолекул и их комплексов, 2) методов интегративного моделирования, позволяющих создавать структурно-динамические, кинетические и статистические модели функционирования биомакромолекул на базе разнородной экспериментальной информации (в том числе геномных данных) с учетом физических взаимодействий между атомами и молекулами, 3) методов структурно-ориентированной системной биологии, позволяющих  интегрировать структурные данные в кинетические модели описывающие работу живых систем, 4) методов машинного обучения, позволяющих создавать предсказательные модели на основе больших объемов многомерных данных.
В ряде вышеозвученных направлений в последнее десятилетие наблюдается существенный прогресс, что закладывает основы для дальнейшего активного развития связанных областей биологии. В частности, в области структурной биологии развиваются методы криоэлектронной микроскопии, применения синхротронного и лазерного рентгеновского излучения (FEL - free electron lasers) для изучения динамики биомолекул. В области методов машинного обучения значительный прогресс связан с развитием как математических алгоритмов, так и с совершенствованием вычислительных возможностей благодаря совершенствованию массивно-параллельных графических процессоров и способов их программирования. Развитие вычислительных технологий также способствует постоянному увеличению возможностей молекулярного моделирования. В области системной и инженерной биологии активно развиваются подходы, связанные с математическим описанием биологических систем по аналогии с различными инженерными системами. 

%\section*{Положения выносимые на защиту}
%\addcontentsline{toc}{section}{Положения выносимые на защиту}
%%положения выносимые на защиту
%Положения выносимые на защиту.

%%%%
%%%%Основные постулаты написания положений в диссертации

%Положения могут содержать следующие элементы: 

%авторские или уточненные автором определения
%научные выводы автора
%основополагающие принципы изученной темы
%классификации и характеристики определенных категорий
%перечни
%предложения
%пути совершенствования объекта изучения и т.д.

%Обычно введение включает в себя 3-6 пунктов положений, рядом с номером пункта необходимо написать краткое содержание рассмотренной и решенной автором задачи. Ниже представлены примеры фраз, с которых они начинаются:

%«Разработаны основные научные выводы»;
%«На защиту выносятся следующие результаты научной деятельности…»;
%«На защиту выносятся следующие новые и содержащие элементы новизны основные идеи»;
%«В ходе работы выявлены факторы, которые влияют на…»;
%«Выявлена взаимосвязь между основными элементами…»;
%«Определена целесообразность внедрения…» и т.д.
%Такие выводы должны быть представлены два раза – в автореферате и, непосредственно, во введении. 
%%%%
%\begin{enumerate}
%гл1
%\item Разработанные методы интегративного моделирования позволяют создавать структурно-динамические модели биомакромолекул и их комплексов на основе наборов разнородных экспериментальных данных.

%\item Предложены комплексные подходы по моделированию структуры и динамики ДНК-белковых комплексов с учетом разнородных экспериментальных данных, в том числе низкого информационного содержания.
%гл2

  %\item Показано, что экспериментальные данные по расщеплению ДНК гидроксильными радикалами могут эффективно использоваться для построения структуры комплексов ДНК-белок. Наличие у комплекса оси симметрии второго порядка помогает установить точное (с точностью до одного нуклеотида) положение ДНК относительно белка в комплексе. Такой подход, в частности, позволяет реконструировать структуры нуклеосом с неизвестным положением ДНК на гистоновом октамере.


  %\item Разработанные методы интегративного моделирования позволяют рассчитывать структурно-динамические модели супрануклеосомной структуры хроматина, описывать эффекты связывания белков хроматина с нуклеосомами.


 % \item Методы молекулярной динамики и диссипативной динамики частиц позволяют устанавливать связь между расположением бета-нитей в амилоидоподобных фибриллах и их крупномасштабными геометрическими характеристиками. Комбинация данных подходов с анализом экспериментальных данных (например, данных ИК и КД спектроскопии, атомно-силовой и электронной микроскопии, дифракции рентгеновских лучей на фибриллах) позволяет построить структурные модели амилоидоподобных фибрилл. 

%\end{enumerate}

% Ниже то, что согласовано с рубиным.
% Концептуально обоснованы и реализованы интегративные подходы создания моделей нуклеосом, комплексов нуклеосом с белками хроматина, амилоидоподобных фибрилл на основе методов атомистического моделирования биомакромолекул методами молекулярной динамики, огрубленного молекулярного моделирования и методов учета разнородных экспериментальных данных рентгеноструктурного анализа, атомно-силовой и электронной микроскопии, футпринтинга ДНК, ИК-, КД-спектроскопии, измерений расстояний между флуоресцентными метками на основе эффекта Ферстеровского резонансного переноса энергии.

%   Методами суперкомпьютерной атомистической молекулярной динамики при моделировании в микросекундном временном диапазоне, методами интегративного моделирования воспроизведены функционально важные крупномасштабные конформационные перестройки структуры ДНК-белковых комплексов (отворот ДНК от октамера гистонов, диффузия гистоновых хвостов вдоль ДНК) и построены модели нуклеосом, комплексов нуклеосом с белками хроматина, амилоидоподобных фибрилл.

\begin{enumerate}

\item	Для построения структурно-динамических моделей сложных биомакромолекулярных комплексов концептуально обосновано применение новых интегративных подходов на основе сочетания методов атомистического и огрубленного молекулярного моделирования, методов учета разнородных экспериментальных данных рентгеноструктурного анализа, атомно-силовой и электронной микроскопии, футпринтинга ДНК, ИК-, КД-спектроскопии, измерений расстояний между флуоресцентными метками на основе эффекта Ферстеровского резонансного переноса энергии.
\item	С использованием разработанного интегративного подхода возможно создание атомистических моделей нуклеосом и комплексов нуклеосом с белками хроматина, при этом учет симметрии белковых комплексов позволяет значительно повысить точность построения молекулярных моделей на основе данных футпринтинга ДНК.
\item	Интегративное моделирование позволяет воспроизвести на атомистическом уровне функциональную динамику нуклеосом, важную с точки зрения эпигенетической регуляции функционирования генома, включая крупномасштабные конформационные перестройки структуры ДНК-белковых комплексов (углы входа-выхода ДНК в нуклеосоме, диффузия гистоновых хвостов вдоль ДНК), а также позволяет обнаружить новые моды динамической подвижности, связанные с изменением конформации ДНК, перестройкой взаимодействий гистоновых хвостов, деформацией глобулярных доменов гистонов.
\item	Разработанный подход интегративного моделирования применим для построения молекулярных моделей амилоидоподобных фибрилл, реконструкции укладки пептидов в фибриллах и установления связи между морфологией фибриллы и межмолекулярной укладкой пептидов.
\end{enumerate}


