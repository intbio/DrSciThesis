\section{Методы расчета свободной энергии взаимодействия с растворителем в молекулярном моделировании}

Предположим, что молекулярная система характеризуется некоторым Гамильтонианом, зависящим от параметра: $H(\lambda)$. \textit{Будем говорить, что каждому значению $\lambda$ соответствует некоторое $\lambda$-состояние системы.} Каким образом возможно рассчитать разницу в свободной энергии меду двумя $\lambda$-состояниями?\\

Начиная с 1960-х годов для практической реализации в численных экспериментах широкое применение получили такие методы, как \textit{метод термодинамического интегрирования}, \textit{метод возмущения свободной энергии}, \textit{метод Беннетта}. С появлением работ  Джарзинского, Крукса, Шёртса и др. \cite{jarzinski,crooks,stanford_ben} выявилась более глубокая связь между этими методами. Соотношения неравновесной статистической физики, выведенные Джарзинским (Jarzinsky equality) и Круксом (transient fluctuation theorem) предлагают более общий подход к решению задачи отыскания свободной энергии через неравновесные процессы. При этом, упомянутые ранее методы являются частными случаями в этом общем подходе, но могут также быть получены независимыми способами.
В данной работе нам будут интересны эти три метода, так как они имеют широкое практическое применение.
\subsection{Метод термодинамического интегрирования}
Метод ТИ может быть получен из простых соображений. Свободная энергия Гиббса системы задаётся в общем случае выражением \ref{gibbsFE}.
\begin{equation}
G=-kT\ln\int \exp\left(-\frac{H(V,\vec{p},\vec{q},\lambda)+PV}{kT}\right)d\vec{p}d\vec{q}dV
\label{gibbsFE}
\end{equation}
Для метода ТИ основным является равенство \ref{TI_equal}.
\begin{equation}
\frac{\partial G}{\partial \lambda}=\left\langle \frac{\partial H}{\partial \lambda} \right\rangle
\label{TI_equal}
\end{equation}
И разница свободной энергии между разными $\lambda$-состояниями даётся выражением \ref{TI}.
\begin{equation}
\Delta G=\int \frac{\partial G}{\partial \lambda} d\lambda
\label{TI}
\end{equation}

Метод достаточно эффективен и часто применяется. Однако, необходимость интегрирования функции $\frac{\partial G}{\partial \lambda}$ требует достаточно большого количества вычислений и вносит дополнительную ошибку численного интегрирования, которую трудно оценить при ограниченном наборе известных значений функции.

\subsection{Метод возмущения свободной энергии} 
В методе возмущения свободной энергии (Free energy perturbation - FEP) разница свободной энергии между разными $\lambda$-состояниями может быть записана в виде \ref{fep_equal}, что и определяет метод отыскания разницы свободной энергии между двумя состояниями.
\begin{equation}
\Delta G=-kT \ln{\frac{\int dV\vec{p}d\vec{q}\exp \left \{ -\frac{H(V,\vec{p},\vec{q},\lambda_1)+PV}{kT} \right \}}{\int dV\vec{p}d\vec{q}\exp \left \{ -\frac{H(V,\vec{p},\vec{q},\lambda_0)+PV}{kT} \right \}}}=-kT
 \ln{<\exp{\frac{H_0-H_1}{kT}}>_0}
\label{fep_equal}	
\end{equation}
где $H_0=H(V,\vec{p},\vec{q},\lambda_0)$, $H_1=H(V,\vec{p},\vec{q},\lambda_1)$, усреднение ведётся по NPT-ансамблю соответствующему состоянию системы с $\lambda=\lambda_0$.

\subsection{Метод Беннетта}
Метод Беннетта (Bennett acceptance ratio method) позволяет наиболее эффективным образом комбинировать данные, получаемые при прямом и обратном процессе. Иными словами, мы переводим систему из состояния с $\lambda=\lambda_0$ в состояние с $\lambda=\lambda_1$ и при этом получаем некоторое значение работы $W_F$. Однако возможен и обратный процесс перевода системы из состояния $\lambda_1$ в состояние $\lambda_0$ с работой $W_R$. В 1976 году метод был предложен Беннеттом \cite{bennet} как обобщение метода FEP. Однако в настоящее время стало ясно, что таким образом можно обобщить практически любые методы расчёта свободной энергии. Этот \textit{обобщённый метод Беннетта} основывается на соотношениях Крукса и позволяет построить наилучшую статистику (оценочную функцию максимального правдоподобия) для оценки значения свободной энергии по наборам параметров $W_F$ и $W_R$ (см. \cite{stanford_ben}).\\

В простейшем случае как обобщение метода FEP метод Беннета (оригинальная работа \cite{bennet}, 1976 год) имеет следующий вид.
Итак, допустим, что у нас есть набор работ по бесконечно быстрой трансформации системы из состояния $\lambda_0$ в состояние $\lambda_1$ - ${W_F}$, и обратно - ${W_R}$. Метод Беннета задаёт оценочную функцию, которая определяет значение разницы свободной энергии с минимальной дисперсией среди всех оценочных функций.
Оценочная функция задаётся неявно, как решение уравнения \ref{beneq} по параметру $C$.

\begin{equation}
\label{beneq}
\sum _m^{N_R} f(W_R^m+C)=\sum _n^{N_F} f(W_F^n-C)
\end{equation}
В уравнении (\ref{beneq}) $f$-функция Ферми-Дирака:
$$
f(x)\equiv \frac{1}{1+\exp{\beta x}}
$$
$N_F$ и $N_R$ - соответственно количество полученных замеров в прямом и обратном направлениях, $\beta=1/kT$. Можно показать, что наилучший результат всегда достигается при $N_F\approx N_R$. Более того, метод Беннета имеет преимущество перед простым методом FEP. Так, более эффективным является осуществление выборки значений работы в прямом и обратном направлениях, и затем применение метода Беннета, чем осуществление в два раза большей выборки в одном направлении и применении метода FEP.

\subsection{Методы расширенных ансамблей}
Кроме трёх вышеприведённых методов развиваются и более современные методы расширенных ансамблей. В этих методах происходит одновременное моделирование систем при различных значениях параметра $\lambda$ и между этими состояниями осуществляются переходы, как в методе Монте-Карло. Хорошим примером адаптивного метода расширенных ансамблей является работа \cite{lyubartsev_2004}. Однако, на данный момент программное обеспечение, реализующее такие методы, мало распространено.
