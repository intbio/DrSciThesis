\chapter*{Список сокращений и условных обозначений}             % Заголовок
\addcontentsline{toc}{chapter}{Список сокращений и условных обозначений}  % Добавляем его в оглавление

\begin{description}[align=right,leftmargin=3.5cm]
\item[АСМ]  атомно-силовая микроскопия
\item[ДДЧ]  диссипативная динамика частиц 
\item[ДМС]  диметилсульфат
\item[криоЭМ]  криоэлектронная микроскопия
\item[МД]  молекулярная динамика
\item[ПААГЭ]  полиакриламидный гель электрофорез
\item[ПГУ]   периодические граничные условия
\item[ПТМ]   пост-трансляционные модификации (белков)
\item[ПО]   программное обеспечение
\item[ПЭГ]  полиэтиленгликоль
\item[п.н.]   пара нуклеотидов
\item[РНКП]  РНК-полимераза
\item[РСА]  рентгено-структурный анализ
\item[СКН] система координат нуклеосомы
\item[ТАД] топологически ассоциированный домен
\item[ЯМР-спектроскопия]  спектроскопия ядерного магнитного резонанса
\item[AL-PAR]  однослойная фибрилла, основанная на параллельном расположении $\beta$-нитей
\item[DL-AP]  двухслойная фибрилла, основанная на антипараллельном расположении $\beta$-нитей
\item[DL-PAR]  двухслойная фибрилла, основанная на параллельном расположении $\beta$-нитей
\item[DPD] dissipative particle dynamics, диссипативная динамика частиц 
\item[FRET]   Forster resonance energy transfer, Ферстеровский перенос энергии 
\item[HRF]  hydroxil-radical footprinting, метод футпринтинга ДНК на онсове гидроксильных радикалов
\item[H-SASA] hydrogen atoms solvent accessible surface area, поверхность, доступная растворителю атомов водорода дезоксирибозы в нуклеотиде 
\item[NCP]  ядро нуклеосомы, нуклеосомный кор (nucleosome core particle)
\item[Pol II] РНК-полимераза II 
\item[REMD]  replica exchange molecular dynamics
\item[RMSD]  root-mean-squared-deviation, среднеквадратичное отклонение
\item[SASA]  solvent accessible surface area, поверхность, доступная растворителю
\item[SHL]  super helix location, позиция на суперспирали нуклеосомной ДНК
\item[SL-AP]  однослойная фибрилла, основанная на антипараллельном расположении $\beta$-нитей
\item[spFRET]  single particle Forster resonance energy transfer, Ферстеровский перенос энергии в экспериментах на единичных частицах
\item[ZLS] zero-loop-stabilizing sequence, последовательность стабилизирующая нулевую петлю
\end{description}





