К главе 1. 


Задачи, Методы и подходы структурной биологии тесным образом связаны с методами и подходами атомистического моделирования. С одной стороны методы молекулярной механики он или ином виде применяются реконструкции атомистических структур практически во всех основных экспериментальных методах структурной биологии скобка открывается рентгеноструктурный анализ,, криоэлектронная микроскопия и так далее скобка закрывается. к при этом Чем меньше информационное содержание получаемых экспериментальных данных, тем в большей степени конечная модель полагается на параметры атомных взаимодействий используемые молекулярный механических моделях.
С другой стороны бурное развитие вычислительных технологий и совершенствование качества молекулярная механических Flash модели молекулярная структура позволяет it Стремится решать некоторые задачи структурной геологии С минимальным привлечением экспериментальных данных или как говорят in silico. к 
К к Такого рода задачи относится моделирование динамики макромолекулярная структура которые непосредственно важна для понимания функции биологических молекул немаловажную роль играет понимание ландшафта свободной энергии различных конформация биомолекул так как именно на на Варшавке они просто по структуре скрыто понимание функции этих молекул .
К к нашим возможности или к изучению информационного ландшафта структур биомолекул серьезным образом прогрессирование в последние два десятка лет вместе с прогрессом суперкомпьютерных технологий. к к к с применением конвенциональных суперкомпьютерных архитектур методами молекулярной динамики настоящее время можно изучать динамику биомолекул на временах десятков микросекунд. С применением специализированных компьютерных архитектур речь уже идет о миллисекундах. с другой стороны последние годы был достигнут и серьёзный методологические и прогресс планер разработки различных ускоренных методов изучение информационного пространства макромолекул таким подходом стоит отнести отходы расширенных ансамблей методы динамики поющий силы ускоренный динамики динамики. к позволяет до некоторой степени преодолеть различные Кинетический барьер и хлопушки которые встречают её макромолекулярные системы при моделировании на небольших временах в тоже время позволяет ID сделать правильную количественную оценку вероятности различных информационных состояний схожая группа методов частности так называемые методы химического превращения развитые последние 10-20 лет также позволяют оценивать свободные энергии между различными Mi Состава молекулярных систем.
Ко-оп этих методах супер компьютерного атомистического моделирования in silico основанных достаточно базовых принципах иногда называемых абане шоу методами и пойдет речь в этой главе. в том числе будут изложены и некоторые результаты по применению и разработки таких методов автором данные диссертации.