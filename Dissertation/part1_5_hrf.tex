\subsection{Методы футпринтинга ДНК}
Одним из подходов к изучению структуры ДНК и ДНК-белковых комплексов являются методы физического, химического или ферментативного футпринтинга (от англ. footprint - след). Идея методов состоит в обработке ДНК определенными агентами, которые вносят изменения в ее химическую структуру в зависимости от ее конформации, состава или взаимодействия с другими макромолекулами, с последующим анализом продуктов реакции для установления мест химических реакций вдоль ДНК.

К физическим методам относят облучение ДНК ультрафиолетом \cite{becker_ultraviolet_1988} или рентгеном, что приводит к разрыву ДНК в областях не защищенных белком. С помощью секвенирующего гель электрофореза можно определить длины продуктов реакций и следовательно участки взаимодействия с белком. Аналогичный подход может быть осуществлен с использованием генерируемых в реакционной смеси гидроксил радикалов химическими методами. Методам гидроксильного футпринтинга посвящена глава \ref{part5_hrf} данной диссертационной работы. Наиболее популярным методом ферментативного футпринтинга является использование фермента ДНКазы I, обладающего выраженной эндонуклеазной активностью. К плюсам такого подхода относится простота осуществления эксперимента, к минусам большой размер фермента (около 4 нм), что уменьшает детальность футпринтинга, и наличие некоторой специфичности по последовательности (гидролизует ДНК преимущественно около пиримидинов). Еще одним интересным методом химического футпринтинга является футпринтинг с использованием перманганата калия. Перманганат калия связывается с тиминами в одноцепочечной ДНК, окисляет их, что приводит в конченом итоге к разрыву в ДНК. Таким образом можно исследовать локализацию пузырей транскрипции в геноме, нестандартных ДНК структур  \cite{kouzine_permanganates1_2017}. Футпринтинг с использованием диметилсульфата (ДМС) используется также для изучения ДНК-белковых взаимодействий. ДМС индуцирует метилирование остатков гуанина. Два образца обрабатывают пиперидином, чтобы вызвать химическое расщепление остатков гуанина, модифицированного ДМС, с последующим расщеплением рестрикционными ферментами. После маркировки образцы запускают параллельно на геле для визуализации. Отсутствующие полосы в образце, связанном с белком, соответствуют остаткам гуанина, защищенным от модификации посредством взаимодействия \cite{hornstra_vivo_1993}.
