%\begin{turnpage}
\begin{table}
\caption{\label{tab:amino_codes} Список 19 аминокислот (кроме пролина), их сокращённые обозначения и R-H аналоги.}
\begin{threeparttable}
	\begin{tabular}{lccc}
Сокр. & 1-буквенный код &Название& R-H аналог\tnote{a} \\
	\hline
GLY &G&глицин& водород  				\\
LEU &L&лейцин& изобутан 				\\
ILE &I&изолейцин& н-бутан 			\\
VAL &V&валин& пропан       \\
ALA &A&аланин& метан           \\
PHE &F&фенилаланин& толуол     \\
CYS &C&цистеин& мтанэтиол      \\
MET &M&метионин& метилэтилсульфид\\
THR &T&треонин& этанол		\\
SER &S&серин& метанол		  \\
TRP &W&триптофан& 3-метилиндол \\
TYR &Y&тирозин& 4-метилфенол   \\
GLN &Q&глутамин& пропионамид    \\
LYS &K&лизин& н-бутиламин  \\
ASN &N&аспарагин& ацетамид      \\
GLU &E&глутаминовая кислота& пропионовая кислота  \\
HIS &H&гистидин& 4-метилимидазол  \\
ASP &D&аспарагиновая кислота&  уксусная килота         \\
ARG &R&аргинин& N-пропилгуанидин   

%HIS & 4-methylimidazole   &+ & xxx&xxx & xxx&-0.96\\

\end{tabular}
\begin{tablenotes}
\item[a]{Названия R-H аналагов аминокислот, где R-боковая цепь аминокислоты.}
\end{tablenotes}
\end{threeparttable}
\end{table}

	 