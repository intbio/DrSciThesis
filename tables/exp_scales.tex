%\begin{turnpage}
\begin{table}
\caption{\label{tab:exp_scales} Экспериментальные шкалы свободных энергий переноса боковых цепей аминокислот и аналогичных молекул из воды в меннее полярный растворитель при pH 7, выраженные в ккал/моль.}
\begin{threeparttable}
	\begin{tabular}{lccccc|c}
Аминокислота & Тип\tnote{a} & V$>$W\tnote{b} & CH$>$W\tnote{c} & O$>$W \tnote{d}& O$>$W$_{\text{occ}}$\tnote{e}&CH$>$O\tnote{f} \\
	\hline
GLY  & N & 2.39  &0.94  & 0         & 0&0\\
LEU  & N & 2.28  &4.92  & 2.30      &2.40&1.68\\
ILE  & N & 2.15  &4.92  & 2.46      &2.27&1.52\\
VAL  & N & 1.99  &4.04  & 1.66      &1.61&1.44 \\
ALA  & N & 1.94  &1.81  & 0.42      &0.65&0.45 \\
PHE  & A & -0.76 &2.98  & 2.44      &2.86&-0.40\\
CYS  & P & -1.24 &1.28  & 1.39(2.10)&1.17&-1.05\\
MET  & N & -1.48 &2.35  & 1.68      &1.82&-0.27\\
THR  & P & -4.88 &-2.57 &  0.35     &0.90&-3.86\\
SER  & P & -5.06 &-3.40 &  -0.05    &0.69&-4.29\\
TRP  & A & -5.88 &2.33  & 3.07      &3.24&-1.68\\
TYR  & A & -6.11 &-0.14 &  1.31     &1.86&-2.39\\
GLN  & P & -9.38 &-5.54 & -0.30     &0.38&-6.18\\
LYS  & + & -9.52 &-5.55 &  -1.35    &-1.65&-5.14\\
ASN  & P & -9.68 &-6.64 &  -0.79    &0.30&-6.79\\
GLU  & - & -10.24&-6.81 &  -2.35(-0.87)&-2.48&-5.40\\
HIS  &P+\tnote{g} & -10.27&-4.66 & 0.18&-1.18(1.04)&-5.78\\
ASP  & - & -10.95&-8.72 &  -2.46(-1.05)&-2.49&-7.20\\
ARG  & + & -19.92&-14.92& -1.37&-0.66&-14.49



\end{tabular}
\begin{tablenotes}
	\item[a]{Тип аминокилоты: N - неполярная, алифатическая; P - полярная, незаряженная; A - ароматическая; '+'/'-' - положительно/отрицательно заряженная.}
	\item[b]{ Свободные энергии переноса R-H аналогов из воздуха в воду, по \cite{wolf_1981}}
	 \item[c] Свободные энергии переноса из циклогексана в воду по \cite{wolfenden_1988}.
	\item[d] Энергии переноса между октанолом и водой из работы \cite{fauchere_1983}, отредактированная Вимли \cite{wimley_1996}. Оригинальные значения в скобках.
	\item[e] Полуэкспериментальная шкала переноса из октанола в воду по \cite{wimley_1996}.
		\item[f] {Энергии переноса между октанолом и циклогексаном.}
	\item[g] pK гиститдина близко к 7.	Шкалы переноса V>W и CH>W включают поправку на ионизацию. Данные O>W для неионизированной формы, данные O$>$W$_{\text{occ}}$ для обеих форм.

	
	


\end{tablenotes}	

\end{threeparttable}	
\end{table}
%\end{turnpage}
	 