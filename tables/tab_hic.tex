\begin{table}[p]
\caption{Увеличение разрешения Hi-C подобных методов, благодаря использованию менее специфичных рестриктаз.}
	\label{tab:p1:hic}	
	\begin{tabularx}{\textwidth} { 
  | >{\raggedright\arraybackslash}X 
  | >{\centering\arraybackslash}X 
  | >{\raggedleft\arraybackslash}X 
  | >{\raggedleft\arraybackslash}X |}
  \hline
Протокол Hi-C & Рестриктазы & Сайты рестрикции & Размер сайта рестрикции\\
	\hline

Классический (Lieberman-Aiden et al., 2009) & HindIII, NcoI & AAGCTT, CCATGG & 6 пн \\
\hline
Sexton et al., 2012, Rao et al., 2014 & Dpn-II & GATC & 4 пн \\
\hline
COLA (Darrow et al., 2016) & CviJI & RCGY, R=A/G, Y=C/G & 3 пн \\
\hline
Micro-C (Hsieh et al., 2016) & MNase & Отсутствует. Эндо/экзо нуклеаза, почти полностью уничтожает линкерные участки ДНК &  - \\
\hline
in situ DNAse Hi-C (Ramani et al., 2016) & DNAse I & Отсутствует. Режет ДНК на тетранукеотиды & -  \\
\hline
\end{tabularx}
	 
\end{table}