\begin{table*}[p]

	\begin{tabularx}{\textwidth}{XXXX}
Аминокислота & Вещество аналог боковой цепи & Химическая формула & Сокр. обозначение \\
	\hline

аланин       & метан   					  & CH$_4$      						 &  ALA'  \\
валин     & пропан					    & C$_3$H$_8$ 							 &  VAL'  \\
лейцин & изобутан 					  & C$_4$H$_{10}$			 			 &  LEU'  \\
изолейцин    & н-бутан     					& C$_4$H$_{10}$						 &  ILE'  \\
цистеин   & метанэтиол 					& CH$_3$SH    						 &  CYS'  \\
метионин    & метилэтилсульфид  & C$_2$H$_5$SCH$_3$  			 &  MET'  \\
серин       & метанол				& CH$_3$OH        			   &  SER'  \\
треонин     & этанол								& C$_2$H$_5$OH      			 &  THR'  \\
тирозин& п-крезол						& C$_3$C$_6$H$_4$OH		 	   &  TYR'  \\
фенилаланин & толуол								& C$_6$H$_5$CH$_3$   			 &  PHE'  \\
триптофан    & 3-метилиндол 				& C$_6$H$_4$C$_2$HNHCH$_3$ &  TRP'  \\
аспарагин    & ацетамид					& CH$_3$CONH$_2$					 &  ASN'  \\
глутамин     & пропионамид					& C$_2$H$_5$CONH$_2$       &  GLN'  \\
\end{tabularx}
	
	\caption{Список 13 изученных нейтральных веществ - аналогов боковых цепей аминокислот, из названия, химические формулы и обозначения.}
	\label{molecules}

\end{table*}