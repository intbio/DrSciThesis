%% Согласно ГОСТ Р 7.0.11-2011:
%% 5.3.3 В заключении диссертации излагают итоги выполненного исследования, рекомендации, перспективы дальнейшей разработки темы.
%% 9.2.3 В заключении автореферата диссертации излагают итоги данного исследования, рекомендации и перспективы дальнейшей разработки темы.

В настоящей работе автором развиты новые системные подходы для построения структурно-динамических моделей сложных биомакромолекулярных комплексов. Разработанные интегративные методы молекулярного моделирования основаны на сочетании (интеграции) физических моделей взаимодействия молекул с информацией получаемой из различных источников экспериментальных данных. В ходе исследования развиты подходы, (i) позволяющие проводить анализ динамики атомистических структур в микросекундном временном диапазоне, (ii) учитывать различные экспериментальные данные при построении моделей (например, данные экспериментов по футпринтингу ДНК, данные электронной микроскопии, данные FRET-микроскопии, данные рентгеновской дифракции на фибриллах, атомно-силовой микроскопии и т.д.), (iii) сочетать атомистическое и огрубленное представление молекулярных систем (в частности при моделировании ДНК-содержащих комплексов).

Разработанные автором подходы и методы рекомендуется применять для установления структуры и динамической подвижности биомакромолекулярных комплексов, размер и свойства которых затрудняют применение стандартных методов структурной биологии. Практическое применение данных подходов продемонстрировано в диссертации на примере ряда биомакромолекулярных систем, в том числе нуклеосом, комплексов нуклеосом с белками хроматина, амилоидоподобных фибрилл.

По результатам проведенных работ могут быть сделаны следующие выводы:

\begin{enumerate}
  %\item Разработаны подходы и программные решения для мультимасштабного моделирования комплексов ДНК и белков с использованием информации из экспериментов по ДНК футпринтингу, электронной микроскопии, FRET-микроскопии. Комбинированное представление ДНК как в атомистическом, так и в динуклеотидном приближении позволяет проводить быструю крупномасштабную оптимизацию структур и при этом учитывать атомистические детали взаимодействия ДНК с белками.
\item Разработанный системный подход и программные решения для мультимасштабного моделирования комплексов ДНК и белков могут быть интегрированы с достаточно разнородной экспериментальной информацией по ДНК футпринтингу, электронной микроскопией, FRET-микроскопией. Комбинированное представление ДНК как в атомистическом, так и в динуклеотидном приближении позволяет проводить быструю крупномасштабную оптимизацию структур и при этом учитывать атомистические детали взаимодействия ДНК с белками.

  % \item Разработанные подходы суперкомпьютерного моделирования нуклеосом позволили изучить динамические характеристики нуклеосом в микросекундном временном диапазоне. Охарактеризованы крупномасштабные функциональные конформационные изменения - флуктуации линкерной  ДНК, диссоциация ДНК от гистонового октамера, образование дефектов кручения ДНК, переключения конформаций гистоновых хвостов.

  \item Результаты суперкомпьютерного моделирования структуры и динамики нуклеосом позволили в микросекундном временном диапазоне определить параметры крупномасштабных функциональных конформационных изменений: вычислить ансамбль конформаций линкерной ДНК и показать, что на геометрию сегментов линкерной ДНК влияет электростатическое отталкивание, определить характерное время диссоциации первого оборота ДНК от гистонового октамера, составляющее порядка 10 мкс, установить структурные механизмы образования дефектов кручения ДНК, переключения конформаций гистоновых хвостов.

  %\item Cовременные методы суперкомпьютерной атомистической молекулярной динамики позволяют рассчитывать для малых молекул на основе заданной модели атом-атомных взаимодействий термодинамические параметры гидратации (свободную энергию гидратации и адсорбции) с высокой статистической точностью (ошибка 1 kT и меньше в зависимости от длины расчета). Расчет профилей свободной энергии боковых цепей аминокислот вблизи поверхности воды указывает на наличие минимума свободной энергии на поверхности воды (вблизи участка, где плотность равна половине от объемной плотности). 

  

  \item Разработанные методы обработки экспериментальных данных по расщеплению ДНК гидроксильными радикалами (футпринтинга ДНК) позволяют вычислить вероятность расщепления ДНК для каждого нуклеотида и использовать ее в методах интегративного моделирования. Показано, что профили расщепления ДНК гидроксильными радикалами в нуклеосоме мало зависят от последовательности ДНК, и определяются в основном позиционированием ДНК. Наличие оси псевдосимметрии второго порядка в нуклеосоме позволило предложить алгоритм точного определения положения ДНК в нуклеосоме по данным футпринтинга для двух цепей ДНК.

  \item  На основе разработанных методов интегративного моделирования нуклеосом и их комплексов получены модели взаимодействия нуклеосом с рядом белков хроматина. Полученные модели указывают на то, что конформационно-динамический полиморфизм нуклеосом является важным фактором при их взаимодействии с белками хроматина. Структура нуклеосом может серьезным образом изменяется при взаимодействиях с белками хроматина (например, в изученном нами взаимодействии с комплексом FACT). Структура комплексов нуклеосом с белками хроматина также зачастую зависит от деталей их состава и внешних условий (например, продемонстрировано нами при моделировании полиморфизма связывания нуклеосомы с гистоном H1). Структура и ориентация ДНК на нуклеосоме является важным фактором для узнавания белками хроматина, что продемонстрировано в работах по изучению комплексов CENP-C с нуклеосомой.

 \item Разработанные подходы построения моделей амилоидоподобных фибрилл на основе ряда экспериментальных данных (ИК-, КД-спектроскопия, рентгеноструктурный анализ, атомно-силовая микроскопия, окрашивание красителями) с применением методов молекулярной динамики и диссипативной динамики частиц позволяют изучить связь взаимного расположения пептидов в фибриллярных структурах с крупномасштабной морфологией амилоидоподобных фибрилл, что в свою очередь может быть использовано для реконструкции структуры фибриллы по экспериментальным данным о ее морфологии. Амилоидоподобные фибриллы преимущественно закручиваются в левые спирали/суперспирали, однако уровень закрутки и морфология сильно зависят от типа укладки пептидов на межмолекулярном уровне.

\end{enumerate}


