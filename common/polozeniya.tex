%положения выносимые на защиту
%Положения выносимые на защиту.

%%%%
%%%%Основные постулаты написания положений в диссертации

%Положения могут содержать следующие элементы: 

%авторские или уточненные автором определения
%научные выводы автора
%основополагающие принципы изученной темы
%классификации и характеристики определенных категорий
%перечни
%предложения
%пути совершенствования объекта изучения и т.д.

%Обычно введение включает в себя 3-6 пунктов положений, рядом с номером пункта необходимо написать краткое содержание рассмотренной и решенной автором задачи. Ниже представлены примеры фраз, с которых они начинаются:

%«Разработаны основные научные выводы»;
%«На защиту выносятся следующие результаты научной деятельности…»;
%«На защиту выносятся следующие новые и содержащие элементы новизны основные идеи»;
%«В ходе работы выявлены факторы, которые влияют на…»;
%«Выявлена взаимосвязь между основными элементами…»;
%«Определена целесообразность внедрения…» и т.д.
%Такие выводы должны быть представлены два раза – в автореферате и, непосредственно, во введении. 
%%%%
%\begin{enumerate}
%гл1
%\item Разработанные методы интегративного моделирования позволяют создавать структурно-динамические модели биомакромолекул и их комплексов на основе наборов разнородных экспериментальных данных.

%\item Предложены комплексные подходы по моделированию структуры и динамики ДНК-белковых комплексов с учетом разнородных экспериментальных данных, в том числе низкого информационного содержания.
%гл2

  %\item Показано, что экспериментальные данные по расщеплению ДНК гидроксильными радикалами могут эффективно использоваться для построения структуры комплексов ДНК-белок. Наличие у комплекса оси симметрии второго порядка помогает установить точное (с точностью до одного нуклеотида) положение ДНК относительно белка в комплексе. Такой подход, в частности, позволяет реконструировать структуры нуклеосом с неизвестным положением ДНК на гистоновом октамере.


  %\item Разработанные методы интегративного моделирования позволяют рассчитывать структурно-динамические модели супрануклеосомной структуры хроматина, описывать эффекты связывания белков хроматина с нуклеосомами.


 % \item Методы молекулярной динамики и диссипативной динамики частиц позволяют устанавливать связь между расположением бета-нитей в амилоидоподобных фибриллах и их крупномасштабными геометрическими характеристиками. Комбинация данных подходов с анализом экспериментальных данных (например, данных ИК и КД спектроскопии, атомно-силовой и электронной микроскопии, дифракции рентгеновских лучей на фибриллах) позволяет построить структурные модели амилоидоподобных фибрилл. 

%\end{enumerate}

% Ниже то, что согласовано с рубиным.
% Концептуально обоснованы и реализованы интегративные подходы создания моделей нуклеосом, комплексов нуклеосом с белками хроматина, амилоидоподобных фибрилл на основе методов атомистического моделирования биомакромолекул методами молекулярной динамики, огрубленного молекулярного моделирования и методов учета разнородных экспериментальных данных рентгеноструктурного анализа, атомно-силовой и электронной микроскопии, футпринтинга ДНК, ИК-, КД-спектроскопии, измерений расстояний между флуоресцентными метками на основе эффекта Ферстеровского резонансного переноса энергии.

%   Методами суперкомпьютерной атомистической молекулярной динамики при моделировании в микросекундном временном диапазоне, методами интегративного моделирования воспроизведены функционально важные крупномасштабные конформационные перестройки структуры ДНК-белковых комплексов (отворот ДНК от октамера гистонов, диффузия гистоновых хвостов вдоль ДНК) и построены модели нуклеосом, комплексов нуклеосом с белками хроматина, амилоидоподобных фибрилл.

\begin{enumerate}

\item	Для построения структурно-динамических моделей сложных биомакромолекулярных комплексов концептуально обосновано применение новых интегративных подходов на основе сочетания методов атомистического и огрубленного молекулярного моделирования, методов учета разнородных экспериментальных данных рентгеноструктурного анализа, атомно-силовой и электронной микроскопии, футпринтинга ДНК, ИК-, КД-спектроскопии, измерений расстояний между флуоресцентными метками на основе эффекта Ферстеровского резонансного переноса энергии.
\item	С использованием разработанного интегративного подхода возможно создание атомистических моделей нуклеосом и комплексов нуклеосом с белками хроматина, при этом учет симметрии белковых комплексов позволяет значительно повысить точность построения молекулярных моделей на основе данных футпринтинга ДНК.
\item	Интегративное моделирование позволяет воспроизвести на атомистическом уровне функциональную динамику нуклеосом, важную с точки зрения эпигенетической регуляции функционирования генома, включая крупномасштабные конформационные перестройки структуры ДНК-белковых комплексов (углы входа-выхода ДНК в нуклеосоме, диффузия гистоновых хвостов вдоль ДНК), а также позволяет обнаружить новые моды динамической подвижности, связанные с изменением конформации ДНК, перестройкой взаимодействий гистоновых хвостов, деформацией глобулярных доменов гистонов.
\item	Разработанный подход интегративного моделирования применим для построения молекулярных моделей амилоидоподобных фибрилл, реконструкции укладки пептидов в фибриллах и установления связи между морфологией фибриллы и межмолекулярной укладкой пептидов.
\end{enumerate}
