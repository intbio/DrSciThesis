%%% Основные сведения %%%
\newcommand{\thesisAuthorLastName}{Шайтан}
\newcommand{\thesisAuthorOtherNames}{Алексей Константинович}
\newcommand{\thesisAuthorInitials}{А.\,К.}
\newcommand{\thesisAuthor}             % Диссертация, ФИО автора
{%
    \texorpdfstring{% \texorpdfstring takes two arguments and uses the first for (La)TeX and the second for pdf
        \thesisAuthorLastName~\thesisAuthorOtherNames% так будет отображаться на титульном листе или в тексте, где будет использоваться переменная
    }{%
        \thesisAuthorLastName, \thesisAuthorOtherNames% эта запись для свойств pdf-файла. В таком виде, если pdf будет обработан программами для сбора библиографических сведений, будет правильно представлена фамилия.
    }
}
\newcommand{\thesisAuthorShort}        % Диссертация, ФИО автора инициалами
{\thesisAuthorInitials~\thesisAuthorLastName}
\newcommand{\thesisUdk}                % Диссертация, УДК
{577.3}
\newcommand{\thesisTitle}              % Диссертация, название
{Интегративное моделирование структуры и динамики биомакромолекулярных комплексов}
\newcommand{\thesisSpecialtyNumber}    % Диссертация, специальность, номер
{03.01.09}
\newcommand{\thesisSpecialtyTitle}     % Диссертация, специальность, название (название взято с сайта ВАК для примера)
{Математическая биология, биоинформатика}
%% \newcommand{\thesisSpecialtyTwoNumber} % Диссертация, вторая специальность, номер
%% {\fixme{XX.XX.XX}}
%% \newcommand{\thesisSpecialtyTwoTitle}  % Диссертация, вторая специальность, название
%% {\fixme{Теория и~методика физического воспитания, спортивной тренировки,
%% оздоровительной и~адаптивной физической культуры}}
\newcommand{\thesisDegree}             % Диссертация, ученая степень
{доктора физико-математических наук}
\newcommand{\thesisDegreeShort}        % Диссертация, ученая степень, краткая запись
{докт. физ.-мат. наук}
\newcommand{\thesisCity}               % Диссертация, город написания диссертации
{Москва}
\newcommand{\thesisYear}               % Диссертация, год написания диссертации
{\the\year}
\newcommand{\thesisOrganization}       % Диссертация, организация
{МОСКОВСКИЙ ГОСУДАРСТВЕННЫЙ УНИВЕРСИТЕТ \\ имени М.В.ЛОМОНОСОВА\\ БИОЛОГИЧЕСКИЙ ФАКУЛЬТЕТ}
\newcommand{\thesisOrganizationShort}  % Диссертация, краткое название организации для доклада
{МГУ им. М.В.Ломоносова}

\newcommand{\thesisInOrganization}     % Диссертация, организация в предложном падеже: Работа выполнена в ...
{кафедре биоинженерии биологического факультета МГУ имени М.В.Ломоносова}

%% \newcommand{\supervisorDead}{}           % Рисовать рамку вокруг фамилии
\newcommand{\supervisorFio}              % Научный руководитель, ФИО
{Кирпичников Михаил Петрович}
\newcommand{\supervisorRegalia}          % Научный руководитель, регалии
{д-р~биол. наук, академик РАН}
\newcommand{\supervisorFioShort}         % Научный руководитель, ФИО
{М.\,П.~Кирпичников}
\newcommand{\supervisorRegaliaShort}     % Научный руководитель, регалии
{академик РАН,~д.б.н.}

%% \newcommand{\supervisorTwoDead}{}        % Рисовать рамку вокруг фамилии
%% \newcommand{\supervisorTwoFio}           % Второй научный руководитель, ФИО
%% {\fixme{Фамилия Имя Отчество}}
%% \newcommand{\supervisorTwoRegalia}       % Второй научный руководитель, регалии
%% {\fixme{уч. степень, уч. звание}}
%% \newcommand{\supervisorTwoFioShort}      % Второй научный руководитель, ФИО
%% {\fixme{И.\,О.~Фамилия}}
%% \newcommand{\supervisorTwoRegaliaShort}  % Второй научный руководитель, регалии
%% {\fixme{уч.~ст.,~уч.~зв.}}

\newcommand{\opponentOneFio}           % Оппонент 1, ФИО
{Разин Сергей Владимирович}
\newcommand{\opponentOneRegalia}       % Оппонент 1, регалии
{доктор биологических наук, член-корреспондент РАН, профессор}
\newcommand{\opponentOneJobPlace}      % Оппонент 1, место работы
{заведующий кафедрой молекулярной биологии биологического факультета МГУ имени М.В.Ломоносова, главный научный сотрудник, заведующий лабораторией структурно-функциональной организации хромосом Института биологии гена РАН}
\newcommand{\opponentOneJobPost}       % Оппонент 1, должность
{}

\newcommand{\opponentTwoFio}           % Оппонент 2, ФИО
{Ефремов Роман Гербертович}
\newcommand{\opponentTwoRegalia}       % Оппонент 2, регалии
{доктор физико-математических наук, профессор}
\newcommand{\opponentTwoJobPlace}      % Оппонент 2, место работы
{главный научный сотрудник, заведующий лабораторией моделирования биомолекулярных систем Института биоорганической химии им. академиков М.М. Шемякина и Ю.А. Овчинникова РАН}
\newcommand{\opponentTwoJobPost}       % Оппонент 2, должность
{}

 \newcommand{\opponentThreeFio}         % Оппонент 3, ФИО
 {Бриллиантов Николай Васильевич}
 \newcommand{\opponentThreeRegalia}     % Оппонент 3, регалии
 {доктор физико-математических наук}
 \newcommand{\opponentThreeJobPlace}    % Оппонент 3, место работы
 {директор центра по научным и инженерным вычислительным технологиям для задач с большими массивами данных, профессор Сколковского института науки и технологий}
 \newcommand{\opponentThreeJobPost}     % Оппонент 3, должность
 {}

%\newcommand{\leadingOrganizationTitle} % Ведущая организация, дополнительные строки. Удалить, чтобы не отображать в автореферате
%{\fixme{Федеральное государственное бюджетное образовательное учреждение высшего
%профессионального образования с~длинным длинным длинным длинным названием}}

\newcommand{\defenseDate}              % Защита, дата
%{{\_}{\_} {\_}{\_}{\_}{\_}{\_}{\_} 2021~г.~в~{\_}{\_} часов}
{15 апреля 2021~г.~в~14 часов}
\newcommand{\defenseCouncilNumber}     % Защита, номер диссертационного совета
{МГУ.03.02}
\newcommand{\defenseCouncilTitle}      % Защита, учреждение диссертационного совета
{Московского государственного университета имени М.В.Ломоносова}
\newcommand{\defenseCouncilAddress}    % Защита, адрес учреждение диссертационного совета
{119234, Москва, Ленинские горы, МГУ, дом 1, стр. 12, Биологический факультет, ауд. М1}
%\newcommand{\defenseCouncilPhone}      % Телефон для справок
%{\fixme{+7~(0000)~00-00-00}}

\newcommand{\defenseSecretaryFio}      % Секретарь диссертационного совета, ФИО
{Страховская Марина Глебовна}
\newcommand{\defenseSecretaryRegalia}  % Секретарь диссертационного совета, регалии
{д-р~биол. наук}          % Для сокращений есть ГОСТы, например: ГОСТ Р 7.0.12-2011 + http://base.garant.ru/179724/#block_30000

\newcommand{\synopsisLibrary}          % Автореферат, название библиотеки
{Диссертация находится на хранении в отделе диссертаций научной
библиотеки МГУ имени М.В. Ломоносова (Ломоносовский просп., д. 27). Со
сведениями о регистрации участия в защите в удаленном интерактивном
режиме и с диссертацией в электронном виде можно ознакомиться, перейдя
на страницу диссертационного совета по ссылкам:
\url{ https://istina.msu.ru/dissertation_councils/councils/32241428/} \url{https://www.msu.ru/science/dis-sov-msu.html}.}
\newcommand{\synopsisDate}             % Автореферат, дата рассылки
{9 февраля \the\year~года}

% To avoid conflict with beamer class use \providecommand
\providecommand{\keywords}%            % Ключевые слова для метаданных PDF диссертации и автореферата
{интегративное моделирование, молекулярное моделирование, хроматин, нуклеосомы, амилоидные фибриллы, гидроксильный футпринтинг, молекулярная динамика, свободная энергия, гидратация, адсорбция}
